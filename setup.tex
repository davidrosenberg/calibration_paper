\section{Problem formulation}
\label{sec:formulation}

\subsection{Binary classification}

Let $\mathcal{X} \subseteq \mathbb{R}^d$ and $\mathcal{Y} = \{0, 1\}$. Let $X \in \mathcal{X}$ and $Y \in \mathcal{Y}$ be random variables denoting the input and label, given by an unknown joint distribution $P(X, Y)$.
Suppose we have a model $f : \mathcal{X} \to [0, 1]$ where the output of the model represents the model's confidence that the label is 1.

\begin{definition}[Perfect calibration]
$f : \mathcal{X} \to [0, 1]$ is perfectly calibrated if $p = E[Y \; | \; f(X) = p]$ for all $p \in [0, 1]$.
\end{definition}

\begin{definition}[Calibration error]
For $p \geq 1$, the $\ell_p$ calibration error of $f : \mathcal{X} \to [0, 1]$ is given by:
\[ \ell_p\mbox{-CE}(f) = \Big(E\big[ (f(X) - E[Y \; | \; f(X)])^p \big] \Big)^{1/p} \]
\end{definition}

We primarily focus on the $\ell_2^2$ calibration error, defined as $\ell_2^2\mbox{-CE}(f) = (\ell_2\mbox{-CE}(f))^2$, because of its popularity in the literature~\cite{nguyen2015posterior, hendrycks2019anomaly, kuleshov2015calibrated, hendrycks2019pretraining} and its intimate connection to the Brier score~\cite{murphy1973vector,degroot1983forecasters}.

Calibration alone is not sufficient -- consider an image dataset containing $50\%$ dogs and $50\%$ cats.
If $f$ outputs $0.5$ on all inputs, $f$ is calibrated but not very useful.
We typically want our model to be a predictive as possible, subject to a calibration constraint.
For example, we may wish to maximize the Brier score -- also known as the mean-squared error -- as defined below.

\begin{definition}
The mean-squared error of $f : \mathcal{X} \to [0, 1]$ is given by $\mbox{MSE}(f) = \mathbb{E}[(f(X) - Y)^2]$.
\end{definition}

\subsection{Multi-class classification}

In the multi-class setting, $\mathcal{Y} = [k]$ and we have a model $f : \mathcal{X} \to [0, 1]^k$.

\begin{definition}[Top-label calibration]
$f$ is perfectly top-label calibrated if the top prediction is calibrated:
\[ P\Big(Y = \argmax_{j \in [k]} M(X)_j \; | \; \max_{j \in [k]} M(X)_j = p\Big) \quad \forall p \in [0, 1] \]
\end{definition}

We would often like the model to be calibrated on less likely predictions as well -- imagine that a medical diagnosis system says there is a $50\%$ chance a patient has a benign tumor, a $10\%$ chance she has an aggressive form of cancer, and a $40\%$ chance she has one of a long list of other conditions.

\begin{definition}[Marginal calibration]
$f$ is perfectly marginal-calibrated if the prediction for each label is calibrated:
\[ P\Big(y = j \; | \; f_j(x) = p\Big) = p \quad \forall p \in [0, 1], j \in [k] \]
\end{definition}

Note that prior works~\cite{guo2017calibration, hendrycks2019anomaly, hendrycks2019pretraining} often claim to perform multi-class calibration but only measure top-label calibration.
In the Appendix we discuss other notions of multi-class calibration.

For each of these notions of calibration, we can define associated calibration errors.
Let $P(j)$ denote the probability of class $j$ -- if there is no class balance $P(j) = \frac{1}{k}$ for all $j$.

\begin{definition}[Marginal calibration error]
The $\ell_2^2$ marginal calibration error is:
\[ \ell_2^2(f) = \sum_{j = 1}^k P(j) \mathbb{E}\big[ (M(X)_j - E[Y = j \; | \; M(X)_j])^2 \big] \]
\end{definition}

For notational simplicity, our theory focuses on binary classification.
The methods extend trivially to top-label and marginal calibration, so our experiments focus on multi-class calibration.

% We say $M$ is pooled-calibrated (Kuleshov et al) if $J$ is sampled uniformly at random from $[k]$ and:
% \[ P\Big(Y = J | f_J(x) = p) \Big) \quad \forall p \]
