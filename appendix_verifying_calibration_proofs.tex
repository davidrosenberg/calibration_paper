\newpage

\section{Proofs for section~\ref{sec:verifying_calibration}}
\label{sec:verifying_calibration_appendix}

In this section we prove the finite sample bounds for the plugin and debiased estimators. We follow a very similar structure for both the plugin estimator and the debiased estimators.

We first give a proof for the plugin estimator. At a high level, we decompose the plugin estimator into three terms (Lemma~\ref{lem:plugin_decomp}), and then bound each of these terms. Most of these terms simply involve algebraic manipulation and standard concentration results, except Lemma~\ref{lem:p2_bound} which requires some tricky conditioning.

The debiased estimator decomposes into three terms as well, two of these terms are the same as those in the plugin estimator. Bounding the third term (Lemma~\ref{lem:c3_bound}) is the key to the improved sample complexity of the plugin estimator. The debiased estimator is not completely unbiased. However, with high probability if we condition on the $x_i$s in the evaluation set, each of these error terms is unbiased. We can then use Hoeffding's to concentrate each term near 0. The errors in each bin are then independent which leads to some cancelations of the error terms when we sum them up.

\textbf{We use the following notation simplification} to simplify the theorem statements and proofs:
\[ p_i = P(f(X) = s_i) \]
\[ y_i^* = \mathbb{E}[Y \; | \; f(X) = s_i] \]
\[ e_i = (s_i - y_i^*) \]

Then, if we let ${E^*}^2$ denote the actual $\lsquared$ calibration error, we have:
\[ {E^*}^2 = \sum_{i=1}^b p_i e_i^2 \]

% \begin{lemma}
% Suppose $p_i > \piSmallBound{}$ for all $i$. Then except with probability $\delta$ for all $i$ we have:
% \[ |\hat{p_i} - p_i| < \sqrt{\frac{3}{np_i} \log{\frac{2B}{\delta}}} \]
% % For some $k > 1$, suppose $p_i \geq \frac{1}{kb}$ for all $i$. Then if $n \geq 3kb \log{\frac{2b}{\delta}}$, we have $P(\forall i . \; |p_i - \hat{p_i}| \leq c(n) p_i) \geq 1 - \delta$ where
% \end{lemma}

We begin by noting that $\hat{p_i}$ is close to $p_i$ for all $i$. This is a standard application of either Bernstein's inequality or the multiplicative Chernoff bound.

\begin{lemma}
\label{lem:pi_bound}
Suppose $p_i > \piSmallBound{}$ for all $i$. Then we can define $c(n) < 0.5$ such that except with probability $\delta$ for all $i$ we have:
\[ |\hat{p_i} - p_i| < c(n)p_i := \sqrt{\frac{3}{n \min p_i} \log{\frac{2B}{\delta}}} p_i \]
\end{lemma}

\subsection{Analysis of plugin estimator (proof of Theorem~\ref{thm:final-plugin})}

The following lemma is crucial -- we decompose the plugin estimator into three terms that we can bound separately.

\begin{lemma}[Plugin decomposition]
\label{lem:plugin_decomp}
The plugin estimator satisfies the following decomposition:
\[ \pluginEst{} = \underbrace{\sum_{i=1}^b \hat{p_i}e_i^2}_{(P1)}  - \underbrace{2\sum_{i=1}^b \hat{p_i}e_i(\hat{y_i} - y_i^*)}_{(P2)} + \underbrace{\sum_{i=1}^b \hat{p_i}(\hat{y_i} - y_i^*)^2}_{(P3)} \]
\end{lemma}

\begin{proof}
The proof is by algebra:
\begin{align*}
\pluginEst{} &= \sum_{i=1}^b \hat{p_i}(s_i - \hat{y_i})^2 \\
&= \sum_{i=1}^b \hat{p_i}[e_i - (\hat{y_i} - y_i^*)]^2 \\
&= \underbrace{\sum_{i=1}^b \hat{p_i}e_i^2}_{(P1)}  - \underbrace{2\sum_{i=1}^b \hat{p_i}e_i(\hat{y_i} - y_i^*)}_{(P2)} + \underbrace{\sum_{i=1}^b \hat{p_i}(\hat{y_i} - y_i^*)^2}_{(P3)}
\end{align*}
\end{proof}

We now bound each of these three terms with the following three lemmas. We condition on $|\hat{p_i} - p_i| < c(n)p_i < 0.5p_i$ for all $i$, which holds with high probability from Lemma~\ref{lem:pi_bound}.

\tm{In general, writing proofs is like writing code. lemmas and theorems have type checked inputs and outputs. Inputs are conditiond and definitions, outputs ar conclusions. }
\begin{lemma}
\label{lem:p1_bound}
Let $(P1)$ be as defined in Lemma~\ref{lem:plugin_decomp}.
Suppose $|\hat{p_i} - p_i| < c(n) p_i$ for all $i$. Then
\[ |(P1) - {E^*}^2| \leq c(n) {E^*}^2 \]
\end{lemma}

\begin{proof}
The proof is by algebra. 
\begin{align*}
|(P1) - {E^*}^2| &= | \sum_{i=1}^b \hat{p_i}e_i^2 - \sum_{i=1}^b p_i e_i^2 | \\
&= | \sum_{i=1}^b (\hat{p_i} - p_i) e_i^2 | \\
&\leq \sum_{i=1}^b |(\hat{p_i} - p_i)| e_i^2 \\
&\leq \sum_{i=1}^b c(n) p_i e_i^2 \\
&\leq c(n) \sum_{i=1}^b p_i e_i^2 \\
&\leq c(n) {E^*}^2
\end{align*}
\end{proof}

\begin{lemma}
\label{lem:p2_bound}
Let $(P2)$ be as defined in Lemma~\ref{lem:plugin_decomp}.
Suppose $|\hat{p_i} - p_i| < c(n)p_i < 0.5p_i$ for all $i$. Then with probability $\geq 1 - \delta$:
\[ |(P2)| \leq \sqrt{\frac{2(1+c(n)){E^*}^2}{n} \log{\frac{2}{\delta}}} \]
\end{lemma}

\begin{proof}
Recall that we evaluated our estimators on an independent and identically distributed evaluation set $T_n = \{(x_1, y_1), \dots, (x_n, y_n)\}$. Also, note that since $|\hat{p_i} - p_i| < p_i$, $\hat{p_i} > 0$. Let $Z = (f(x_1), \cdots, f(x_n))$ be a random variable. 

$\hat{y_i}$ simply takes the empirical average of the label values, and is therefore an unbiased estimator of $y_i^*$ even if we condition on $Z$:
\[ \expect[\hat{y_i} - y_i^* \mid Z] = 0 \]

Next we look at the distribution of $\hat{y_i} - y_i^* \mid Z$. For all $(x_j, y_j) \in T_n$, $y_j \in \{0, 1\}$. Additionally, $\{y_j \mid (x_j, y_j) \in T_n\} \mid Z$ is also independently (but not identically) distributed. So by Hoeffding's lemma, $\hat{y_i} - y_i^* \mid Z$ is sub-Gaussian with parameter $\frac{1}{4 \hat{p_i} n}$.

Here, we note that $\hat{p_i}$ is a constant given $Z$.
Then, we get that $\hat{p_i}e_i(\hat{y_i} - y_i^*) \mid Z$ has expected value $0$ and is sub-Gaussian with parameter:
\[ \sigma_i^2 = \hat{p_i}^2 e_i^2 \frac{1}{4 \hat{p_i} n} = \frac{\hat{p_i} e_i^2}{4n} \]
This means that the sum, $(P2)$ has expected value $0$ and is sub-Gaussian with  parameter:
\[ \sigma^2 = 2^2 \sum_{i=1}^B \sigma_i^2 = 4 \sum_{i=1}^B \frac{\hat{p_i} e_i^2}{4n} \leq \frac{(1+c(n)){E^*}^2}{n} \]
By applying the sub-Gaussian tail inequality, we get that with probability at least $1-\delta$,
\[ |(P2)| \leq \sqrt{\frac{2(1+c(n)){E^*}^2}{n} \log{\frac{2}{\delta}}} \]
Since this was true for all $Z$, this is true if we marginalize over $Z$ as well, which completes the proof.
\end{proof}

\begin{lemma}
\label{lem:p3_bound}
Let $(P3)$ be as defined in Lemma~\ref{lem:plugin_decomp}.
Suppose $|\hat{p_i} - p_i| < c(n)p_i < 0.5p_i$ for all $i$. Then with probability $\geq 1 - \delta$:
\[ |(P3)| \leq \frac{B}{2n} \log{\frac{2B}{\delta}} \]
\end{lemma}

\begin{proof}
Fix arbitrary $\hat{p_i}$s satisfying $|\hat{p_i} - p_i| < c(n) p_i < 0.5p_i$. Note that this gives us $\hat{p_i} > 0$.

By Hoeffding's bound, for any fixed $i$, with probability at least $1-\frac{\delta}{b}$:
\[ |\hat{y_i} - y_i^*| \leq \sqrt{\frac{1}{2\hat{p_i}n} \log{\frac{2B}{\delta}}} \]
Applying union bound over $i = 1, \cdots, B$, we get that the above holds \emph{for all $i$} with probability at least $1 - \delta$. Then with probability at least $1 - \delta$, for all $i$:
\[ | \hat{p_i} (\hat{y_i} - y_i^*)^2 | \leq \frac{1}{2n} \log{\frac{2B}{\delta}} \]
Summing over the bins $i = 1, \cdots, B$, we get:
\[ 0 \leq (P3) \leq \frac{B}{2n} \log{\frac{2B}{\delta}}  \]
\end{proof}


Bounding the error of the plugin estimator simply involves combining the bounds for each of the terms, $P1$, $P2$, $P3$.

\begin{restatable}{theorem}{pluginBound}
\label{thm:plugin-bound}
Let $p_i = P(f(X) = s_i)$ and suppose $p_i > \piSmallBound{}$ for all $i$. Let $c(n)$ be defined as:
\[ c(n) = \sqrt{\frac{3}{n \min p_i} \log{\frac{2B}{\delta}}} \]
Then for the plugin estimator, with probability at least $1 - 3\delta$,
\[ | \hat{E_{pl}}^2 - {E^*}^2 | \leq c(n){E^*}^2 + \sqrt{\frac{2(1+c(n)){E^*}^2}{n} \log{\frac{2}{\delta}}} + \frac{B}{2n} \log{\frac{2B}{\delta}} \]
\end{restatable}

\begin{proof}
We have:
\[ |\pluginEst{} - {E^*}^2| \leq |P1 - {E^*}^2| + |P2| + |P3| \]
From Lemma~\ref{lem:pi_bound} we have $|\hat{p_i} - p_i| < c(n)p_i < 0.5p_i$ with probability $\geq 1 - \delta$. Conditioning on this, we combine Lemmas~\ref{lem:p1_bound},~\ref{lem:p2_bound},~\ref{lem:p3_bound} with union bound to get the desired result.
\end{proof}

We then prove the final bound for the plugin estimator, which we recall below.

\begin{finalPlugin}
\finalPluginText{}
\end{finalPlugin}

\begin{proof}
This is now a simple corollary of Theorem~\ref{thm:plugin-bound}. For large enough constant $c$, where $c$ is a constant independent of all the other variables, we choose $n = c\frac{B}{\epsilon^2} \log{\frac{B}{\epsilon^2}}$. Plugging it into the bound of Theorem~\ref{thm:plugin-bound}, we get the desired result, that $\lvert \hat{E}^2 - {E^*}^2 \rvert \leq \frac{1}{2}{E^*}^2$. Notice that the dominating term is term $(P3)$ in Theorem~\ref{thm:plugin-bound}---we will see that the debiased estimator improves on this.

In fact, we can also show that in the worst case the plugin estimator will need at least $O(\frac{B}{\epsilon^2})$ samples to estimate the calibration error. To see this, first note that the bias of the plugin estimator, which comes from term $(P3)$ is at least $\frac{B}{n}$. Furthermore, in the analysis of the debiased estimator we show that the variance of this term is on the order of $O(\frac{\sqrt{B}}{n})$. So if $n < 0.1\frac{B}{\epsilon^2}$ we can consider very large $B$, and use Chebyshev to show that that with high probability the estimation error is larger than $\epsilon^2$. 
% We first provide an intuitive argument, and then a formal proof. When $r = 0.5$, we intuitively want to distinguish between when ${E^*}^2$ is $\epsilon^2$ and $0.5\epsilon^2$. So we want to bound the confidence interval by a constant factor of $\epsilon^2$ in the regime where ${E^*}^2 \approx \epsilon^2$. Substitute ${E^*}^2 = \epsilon^2$ into the bound in Theorem~\ref{thm:plugin-bound}, and write out the condition that $| \pluginEst{} - {E^*}^2 | \leq \frac{\epsilon^2}{4}$. With some simple algebra, we get $n = \Theta(B + \frac{B}{\epsilon})$.

% Now we give the formal proof, which formalizes the above intuition. First, we note that if $n \geq 6B \log{\frac{B}{\delta}}$ then $p_i > \piSmallBound{}$ so this satisfies the requirements to apply Theorem~\ref{thm:plugin-bound}.

% Let $\omega = \frac{1-r}{2}$. Note that $r + \omega = 1 - \omega$. We split into two cases. In the first case, we will assume that ${E^*}^2 \leq r \epsilon^2$. We will show that with probability at least $1 - \delta$, $\pluginEst{} \leq (1-\omega) \epsilon^2$. In the second case, we will assume that ${E^*}^2 > \epsilon^2$. We will show that with probability at least $1 - \delta$, $\pluginEst{} > (1-\omega) \epsilon^2$. This then gives us an easy way to hypothesis test between the two cases. That is, we will say the model is probably calibrated iff $\pluginEst{} \leq (1-\omega) \epsilon^2$.

% \textbf{Case 1}: Suppose  ${E^*}^2 \leq r \epsilon^2$.

% We will show that $| \pluginEst{} - {E^*}^2 | \leq \omega \epsilon^2$ by bounding each of the three term of  $| \pluginEst{} - {E^*}^2 |$ in Theorem~\ref{thm:plugin-bound} by $\frac{\omega}{3}\epsilon^2$.

% For the first term, suppose $n \geq (\frac{54r^2}{\omega^2}\log{\frac{2B}{\delta}})B$ . Then $c(n) {E^*}^2 \leq \frac{\omega}{3}\epsilon^2$.

% For the second term, we first note that from Lemma~\ref{lem:pi_bound}, $c(n) < 1$. So, we have,
% \[ \sqrt{\frac{2(1+c(n)){E^*}^2}{n} \log{\frac{2}{\delta}}} \leq \sqrt{\frac{4{E^*}^2}{n} \log{\frac{2}{\delta}}} \]
% Then, if $n \geq (\frac{36r^2}{\omega^2} \log{\frac{2}{\delta}})\frac{1}{\epsilon^2}$, we have,
% \[ \sqrt{\frac{4{E^*}^2}{n} \log{\frac{2}{\delta}}} \leq \sqrt{\frac{4r^2\epsilon^2}{n} \log{\frac{2}{\delta}}} \leq \frac{\omega}{3}\epsilon^2 \]

% Finally, for the third term, if $n \geq (\frac{3}{2\omega} \log{\frac{2B}{\delta}}) \frac{B}{\epsilon^2}$, then,
% \[ \frac{B}{2n} \log{\frac{2B}{\delta}} \leq \frac{\omega}{3}\epsilon^2 \]
% In total, for constant $\omega$ and ignoring $\log$ factors, this means if $n = \Theta(B + \frac{B}{\epsilon^2})$, then $\hat{E}^2 \leq (1-\omega) \epsilon^2$.

% \textbf{Case 2}: Suppose ${E^*}^2 > \epsilon^2$.

% It suffices to show that $| \pluginEst{} - {E^*}^2 | \leq \omega {E^*}^2$, because then $\pluginEst{} > (1 - \omega) {E^*}^2 \geq (1 - \omega) \epsilon^2$. As in the previous case, we show this by bounding each of the three terms of  $| \pluginEst{} - {E^*}^2 |$ in Theorem~\ref{thm:plugin-bound} by $\frac{\omega}{3}{E^*}^2$. The proof of this case is nearly identical to the previous case and using the fact that $\epsilon^2 \leq {E^*}^2$.

\end{proof}

\subsection{Analysis of debiased estimator (proof of Theorem~\ref{thm:final-ours})}

Next, we bound the error of the debiased estimator. The proof follows along the lines of the plugin estimator. We begin with a decomposition (Lemma~\ref{lem:canceling_decomp}), similar to the decomposition of the plugin estimator. However, one of the terms in the decomposition, $C3$, is different. Lemma~\ref{lem:c3_bound} bounds this term $C3$. The rest of the proof is the same as for the plugin estimator, so we omit the other proofs.

As with the plugin estimator, we have a decomposition for the debiased estimator.

\begin{lemma}[Debiased decomposition]
\label{lem:canceling_decomp}
The debiased estimator satisfies the following decomposition:
\[ \cancelEst{} = \underbrace{\sum_{i=1}^b \hat{p_i}e_i^2}_{(C1)}  - \underbrace{2\sum_{i=1}^b \hat{p_i}e_i(\hat{y_i} - y_i^*)}_{(C2)} + \underbrace{\sum_{i=1}^b \hat{p_i}\Big[ (\hat{y_i} - y_i^*)^2 - \frac{\hat{y_i}(1 - \hat{y_i})}{\hat{p_i}n-1} \Big]}_{(C3)} \]
\end{lemma}

As with the plugin estimator, we bound each of the three terms. Notice that $C1$ and $C2$ are the same as terms $P1$ and $P2$ in the plugin estimator decomposition, so the bounds for those carry over. The next lemma bounds the error in $C3$.

\begin{lemma}
\label{lem:c3_bound}
Let $(C3)$ be as defined in Lemma~\ref{lem:canceling_decomp}.
Suppose $|\hat{p_i} - p_i| < c(n) p_i < 0.5 p_i$ for all $i$. Then with probability $\geq 1 - \delta$:
\[ |(C3)| \leq \frac{3\sqrt{B}}{n} \log{\frac{n}{\delta}} + \frac{\delta}{n} \]
\end{lemma}

\begin{proof}
Let $Z = (f(x_1), \cdots, f(x_n))$ be a random variable. We note that for all $i$, $\hat{p_i}$ is a deterministic function of $Z$. For convenience, define $t_i$ as follows:
\[ t_i = (\hat{y_i} - y_i^*)^2 - \frac{\hat{y_i}(1 - \hat{y_i})}{\hat{p_i}n-1} \]

\textbf{Computing the expectation:} The debiased estimator debiases the plugin estimator. In particular, we briefly explain why $\expect[C3 \mid Z] = 0$. Since $\hat{y_i}$ is the mean of $n\hat{p_i}$ draws of a Bernoulli with parameter $y_i^*$, we have:
\[ \expect[(\hat{y_i} - y_i^*)^2 \mid Z] = \frac{y_i^*(1 - y_i^*)}{n\hat{p_i}} \] 
The term we subtracted is the unbiased estimate of the standard deviation of the samples, so from elementary statistics:
\[ \expect\Big[\frac{\hat{y_i}(1 - \hat{y_i})}{\hat{p_i}n-1} \mid Z\Big] = \frac{y_i^*(1 - y_i^*)}{n\hat{p_i}} \]

% \[ \expect\Big[\frac{\hat{y_i}(1 - \hat{y_i})}{\hat{p_i}n-1} \mid Z\Big] = \frac{\expect[\hat{y_i} \mid Z] - \expect[\hat{y_i}^2 \mid Z]}{\hat{p_i}n-1} \]
% $\expect[\hat{y_i} \mid Z] = \hat{p_i}$ and $\expect[\hat{y_i}^2 \mid Z] = \frac{\hat{p_i}(1 - \hat{p_i})}{n} + \hat{p_i}^2$. Substituting these in gives us:
% \[ \expect\Big[\frac{\hat{y_i}(1 - \hat{y_i})}{\hat{p_i}n-1} \mid Z\Big] = \frac{y_i^*(1 - y_i^*)}{n\hat{p_i}} \]
Which implies that $\expect[C3 \mid Z] = 0$.

\textbf{Bounding each term:} By Hoeffding's bound, for any fixed $i$, we get that with probability at least $1 - \frac{\delta}{n}$: 
\[ |\hat{y_i} - y_i^*| \leq \sqrt{\frac{1}{2\hat{p_i}n} \log{\frac{2n}{\delta}}} \]

Let $E_i$ be the event that this is indeed the case. Condition on $E_i$ holding for all $i$ -- by union bound this happens with probability at least $1 - \delta$.
With some algebra, we then get:
\[ \lvert \hat{p_i}t_i \rvert = \Big\lvert \hat{p_i}\Big[ (\hat{y_i} - y_i^*)^2 - \frac{\hat{y_i}(1 - \hat{y_i})}{\hat{p_i}n-1} \Big] \Big\rvert \leq \frac{3}{2n} \log{\frac{B}{\delta}} \]

\textbf{Concentration:} Next, we analyze the concentration of $T = \big[(C3) \mid Z, \forall i. E_i\big]$ around its mean $\mu$. $\lvert \hat{p_i}t_i \rvert$ is bounded so is sub-Gaussian with parameter:
\[ \sigma_i^2 = \frac{9}{4n^2}\log{\frac{B}{\delta}} \]
Each term $\hat{p_i}t_i$ in the sum is independent, even when conditioned on $Z$.
So $T$ is sub-Gaussian with parameter:
\[ \sigma^2 = \sum_{i=1}^B \sigma_i^2 = \frac{9B}{4n^2}\log{\frac{B}{\delta}} \]
So by the sub-Gaussian tail bound, we have:
\[ \lvert T - \mu \rvert \leq \sqrt{2\sigma^2\log{\frac{1}{\delta}}} \leq \frac{3\sqrt{2}}{2} \frac{\sqrt{B}}{n} \sqrt{\log{\frac{n}{\delta}} \log{\frac{1}{\delta}}} \]
This can be simplified to:
\[ \lvert T - \mu \rvert \leq \frac{3\sqrt{B}}{n} \log{\frac{n}{\delta}} \]

\textbf{Bounding the bias:} Although $\expect[C3 \mid Z] = 0$, conditioning on $E_i$ introduces some bias.
However, we can show this bias is small. First, notice that $|t_i| \leq 1$. The event $E_i$ holds with probability at least $1 - \frac{\delta}{n}$. Then by the law of total expectation, conditioning on $E_i$ shifts the mean by at most $\frac{\delta}{n}$ -- in other words $|\expect[t_i \mid E_i, Z]| \leq \frac{\delta}{n}$.
Summing over $t_i$s, we get:
\[ \lvert \expect[(C3) \mid Z, \forall i. E_i] \rvert \leq \sum_{i=1}^B \hat{p_i} \lvert \expect[t_i \mid E_i, Z] \rvert \leq \frac{\delta}{n} \]

\textbf{Finishing up:} Combining the bias and concentration, we get that with probability at least $1 - 2\delta$:
\[ |(C3)| \leq \frac{3\sqrt{B}}{n} \log{\frac{n}{\delta}} + \frac{\delta}{n}\]

\end{proof}

\tm{where the theorem above is used. It sounds atypical to have a theorem in appendix..}
\tm{it's almost always better to write theorem first and then lemmas.. }
\ak{The main theorem is the main text, should I restate it at the start of this section? In some sense what I presented in the main text is a corollary of the theorem below, but it sounds weird to give a corollary without a theorem in the main text. And this theorem is a bit too unwieldy to put in the main text.}
We combine the bounds for $(C1)$, $(C2)$, $(C3)$, as in Theorem~\ref{thm:plugin-bound}, to bound the estimation error of the debiased estimator.

\begin{restatable}{theorem}{cancelingBound}
\label{thm:our-bound}
% Let $p_i = P(f(X) = s_i)$ and suppose $p_i > \piSmallBound{}$ for all $i$. Let $c(n)$ be as defined in Theorem~\ref{thm:plugin-bound}.
In the same setting as Theorem~\ref{thm:plugin-bound}, for the debiased estimator, with probability at least $1 - 4\delta$,
\[ | \hat{E}^2 - {E^*}^2 | \leq c(n){E^*}^2 + \sqrt{\frac{2(1+c(n)){E^*}^2}{n} \log{\frac{2}{\delta}}} + \frac{3\sqrt{B}}{n} \log{\frac{n}{\delta}} + \frac{\delta}{n}\]
\end{restatable}

We interpret the bound in Theorem~\ref{thm:our-bound} in two regimes. In the first regime, we fix the problem parameters $p_i, {E^*}^2$, and look at what happens as we send $n$ to infinity. In that case, the second term dominates, and we see that the error is approximately proportional to $\frac{1}{\sqrt{n}}$, which is the same as for the plugin estimator. However, in general we do not need to estimate the calibration error extremely finely, and may be satisfied as long as we estimate the calibration error within a constant multiplicative factor. That is, we might only need $n$ to be large enough so that our estimate $\hat{E}^2$ is on the right order, e.g. between $0.5 {E^*}^2$ and $1.5 {E^*}^2$ (where $0.5$ and $1.5$ can be replaced by other constants). In that regime, the third term dominates and the error is approximately proportional to $\frac{\sqrt{B}}{n}$, which is better than for the plugin estimator where it is proportional to $\frac{B}{n}$ (see Theorem~\ref{thm:plugin-bound}). This is captured in the final bound, where the proof closely parallels that of Theorem~\ref{thm:final-plugin}.

\begin{finalCanceling}
\finalCancelingText{}
\end{finalCanceling}
