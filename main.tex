\documentclass{article}

% if you need to pass options to natbib, use, e.g.:
%     \PassOptionsToPackage{numbers, compress}{natbib}
% before loading neurips_2019

% ready for submission
\usepackage{neurips_2019}
\usepackage{amsfonts}
\usepackage{amsthm}
\usepackage{amsmath}
\usepackage{graphicx}
\usepackage{caption}
\usepackage{subcaption}
\usepackage{xcolor}
\usepackage{thmtools, thm-restate}

\newtheorem{theorem}{Theorem}[section]
\newtheorem{lemma}[theorem]{Lemma}
\newtheorem{definition}[theorem]{Definition}
\newtheorem{example}[theorem]{Example}
\newtheorem{proposition}[theorem]{Proposition}
\newtheorem{corollary}[theorem]{Corollary}


% \newcommand{\pl}[1]{}
% \newcommand{\tm}[1]{}
% \newcommand{\ak}[1]{}

\newcommand{\pl}[1]{\textcolor{red}{[PL: #1]}}
\newcommand{\tm}[1]{\textcolor{red}{[TM: #1]}}
\newcommand{\ak}[1]{\textcolor{red}{[AK: #1]}}


\graphicspath{ {images/} }

% to compile a preprint version, e.g., for submission to arXiv, add add the
% [preprint] option:
%     \usepackage[preprint]{neurips_2019}

% to compile a camera-ready version, add the [final] option, e.g.:
    %  \usepackage[final]{neurips_2019}

% to avoid loading the natbib package, add option nonatbib:
%     \usepackage[nonatbib]{neurips_2019}

\usepackage[utf8]{inputenc} % allow utf-8 input
\usepackage[T1]{fontenc}    % use 8-bit T1 fonts
\usepackage{hyperref}       % hyperlinks
\usepackage{url}            % simple URL typesetting
\usepackage{booktabs}       % professional-quality tables
\usepackage{amsfonts}       % blackboard math symbols
\usepackage{nicefrac}       % compact symbols for 1/2, etc.
\usepackage{microtype}      % microtypography

\newcommand{\argmax}{\mathop{\mbox{argmax}}}
\newcommand{\argmin}{\mathop{\mbox{argmin}}}
\newcommand{\argsup}{\mathop{\mbox{argsup}}}
\newcommand{\bins}{\mathcal{B}}

\title{Variance Reduced Uncertainty Calibration}

% The \author macro works with any number of authors. There are two commands
% used to separate the names and addresses of multiple authors: \And and \AND.
%
% Using \And between authors leaves it to LaTeX to determine where to break the
% lines. Using \AND forces a line break at that point. So, if LaTeX puts 3 of 4
% authors names on the first line, and the last on the second line, try using
% \AND instead of \And before the third author name.

\author{%
  David S.~Hippocampus\thanks{Use footnote for providing further information
    about author (webpage, alternative address)---\emph{not} for acknowledging
    funding agencies.} \\
  Department of Computer Science\\
  Cranberry-Lemon University\\
  Pittsburgh, PA 15213 \\
  \texttt{hippo@cs.cranberry-lemon.edu} \\
  % examples of more authors
  % \And
  % Coauthor \\
  % Affiliation \\
  % Address \\
  % \texttt{email} \\
}

\begin{document}

\maketitle

\begin{abstract}
In applications like weather forecasting and personalized medicine we require that models output calibrated probability estimates - estimates representative of the true likelihood of an event. We discover that previous calibration methods like Platt scaling and temperature scaling are (i) less calibrated than reported and (ii) we do not know how miscalibrated they are, because prior work only gives lower bounds for their calibration error. Other methods like histogram binning are sample inefficient -- the number of samples required scales linearly with the number of model outputs. To solve these problems, we introduce the variance-reduced calibrator. The variance-reduced calibrator first fits a parametric function -- which acts like a baseline for variance reduction -- and then bins the function values to actually ensure calibration. We then turn to an important question: how do we verify calibration? We introduce a new estimator for the calibration error that requires fewer samples. We prove finite sample guarantees for all our results, and validate our theory with multi-class calibration experiments on CIFAR-10, where we get an X\% lower calibration error than histogram binning.

% In many applications, outputting calibrated probability estimates -- probability estimates representative of the true frequency of an event happening -- is as important as obtaining high accuracy. We show, both in theory and practice, that past work on calibration underestimates the calibration error of models, for example by over $50\%$ on CIFAR-10. We introduce a verified-calibration wrapper -- given an uncalibrated model and a small amount of data, our method will return a calibrated model and an upper bound on its calibration error. The overarching approach is simple -- \pl{use parallel structure: do a, do b, do c} we post-process with methods like Platt Scaling, then discretize the outputs of the model in a way that ensures the calibration error is measurable, and we introduce a more sample-efficient estimator to measure the calibration error. \pl{somehow figure out how to put the core intuition in} We prove theorems showing that discretization incurs minimal overhead in sample complexity, and has little impact on model quality given by the Brier score. These are backed by experiments on calibrating neural networks on vision and text datasets \pl{what's result?}. Finally, we show issues with past work \pl{careful of tone} on multi-class calibration, and conclude that multi-class calibration is a major open problem. Anonymized code is available on GitHub%
\end{abstract}

\section{Introduction}

% Applications like personalized medicine, meteorological forecasting, and natural language processing \pl{NLP is so broad...anything could go here; either motivate based on fields that already use calibration (good to stand on shoulders of giants) or say something substantive about why calibration is needed (danger of being too preachy (danger of being too preachy))} require that classifiers provide calibrated confidence measures in addition to their predictions~\cite{jiang2012calibrating, brocker2009decomposition, nguyen2015posterior}.
% In other words, the probability that a system outputs for an event should reflect the true frequency of that event: if an automated diagnosis system says 1,000 patient have cancer with probability 0.1, approximately 100 of them should indeed have cancer.
The probability that a system outputs for an event should reflect the true frequency of that event: if an automated diagnosis system says 1,000 patients have cancer with probability 0.1, approximately 100 of them should indeed have cancer.
In this case we say the model is uncertainty calibrated. The importance of this notion of calibration has been emphasized in personalized medicine~\cite{jiang2012calibrating}, meteorological forecasting~\cite{brocker2009decomposition} and natural language processing applications~\cite{nguyen2015posterior}.
As most machine learning models, including neural networks, do not output calibrated probabilities out of the box~\cite{guo2017calibration, zadrozny2001calibrated} recalibration methods take the output of an uncalibrated model, and transform it into a calibrated probability.
\emph{Scaling} approaches for recalibration include Platt scaling~\cite{platt1999probabilistic}, isotonic regression~\cite{zadrozny2002transforming}, and temperature scaling~\cite{guo2017calibration}. These methods are widely used, but do they actually produce calibrated probabilities?
% , and speech recognition~\cite{dong2011calibration}
% These methods use additional recalibration data to fit a simple function on top of the original model outputs.

% Important, check that it's not too similar to "On calibration of modern neural networks" paper.
% In many applications, classification models must not only be accurate, but should indicate when they may be incorrect. For example, in automated healthcare applications, control should be passed on to human doctors when the confidence of a disease diagnosis prediction is low. Specifically, classifiers should provide a \emph{calibrated} confidence measure in addition to its prediction. In other words, the probability that a system outputs for an event should reflect the true frequency of that event: of the times that a system says a patient has cancer with probability 0.3, 30\% of the time, the patient should indeed have cancer.
% Typically, complex models like neural networks do not output calibrated probabilities.
% Instead, recalibration methods take the output of an uncalibrated model, and transform it into a calibrated probability.
% Popular recalibration approaches include Platt scaling and temperature scaling.

% Recent advances in machine learning have dramatically increased predictive accuracy. Machine learning methods are now entrusted with making decisions in applications ranging from weather forecasting to medical diagnosis \pl{seems a bit naive...}. In many of these settings, classification models must not only be accurate, but should indicate when they may be incorrect. For example, in automated healthcare applications, control should be passed on to human doctors when the confidence of a disease diagnosis prediction is low. Specifically, classifiers should provide a \emph{calibrated} confidence measure in addition to its prediction. In other words, the probability that a system outputs for an event should reflect the true frequency of that event: of the times that a system says that it will rain with probability 0.3,  30\% of the time, it should rain.
% \pl{need citations; is rain prediction really driven by "modern ML"?}\tm{It also seems unncessary to introduce the rain example --- it seems to be a good introduction to people who didn't know it before, but not something to write in the first para of a paper?}

%  \pl{I'd try to combine with first paragraph so that it's all background}
% \pl{introduce calibration error as a central concern first;
% methods try to calibrate, but do they actually achieve it?
% }

\emph{We discover that these methods are less calibrated than reported.} Estimating the calibration of models can be very challenging, \tm{maybe "the calibration of models" -> "the calibration of models with continuous outputs"?}so past work approximates a model's calibration error using a finite set of bins. We show that by using more bins, we can uncover a higher calibration error for models on CIFAR-10 and ImageNet (Figure~\ref{fig:lower_bounds}). This is a fundamental limitation with approaches that output a continuous range of probabilities -- we show that their true calibration error may never be measurable with a finite number of bins (Example~\ref{ex:continuous-not-calibrated}).

An alternative approach, histogram binning~\cite{zadrozny2001calibrated}, outputs probabilities from a finite set.
Histogram binning can produce a model that is calibrated, and unlike scaling methods we can measure its calibration error, but it can be sample inefficient.
In particular, the number of samples required to calibrate scales linearly with the number of model outputs $B$~\cite{naeini2014binary}, which can be large particularly in the multi-class setting where $B$ typically scales with the number of classes.
Recalibration sample efficiency is crucial -- we often want to recalibrate our models in the presence of domain shift~\cite{hendrycks2019anomaly} or recalibrate a model trained on simulated data, and may have access to only a small labeled dataset from the target domain.
\pl{do they actually do calibration there? if not, that's a weird cite}

To get the sample efficiency of Platt scaling and the verification guarantees of histogram binning, \emph{we propose the variance-reduced calibrator} (Figure~\ref{fig:var_red_binning}).
As with other recalibration methods, we begin with a recalibration dataset $\{(z_1, y_1), ..., (z_n, y_n)\}$, where $z_i$ represents the uncalibrated model output and $y_i$ the ground truth label.
In the binary classification setting, $z_i \in [0, 1]$ and $y_i \in \{0, 1\}$.
We fit a simple function $g \in \mathcal{G}$ to the recalibration dataset.
We then bin the input space so that an equal number of $g(z_i)$ land in each bin.
In each bin, we output the average of the $g(z_i)$ values in that bin -- these are the gray circles in Figure~\ref{fig:var_red_binning}.
Binning ensures that the output probabilities are from a finite set, so we can check if it is calibrated.
In contrast, histogram binning takes the average of the $y_i$ values in each bin (Figure~\ref{fig:hist_binning}).
The motivation behind our method is that the $g(z_i)$ values in each bin are in a narrower range than the $y_i$ values -- when we take the average we incur less of an estimation error.
If $\mathcal{G}$ is well chosen, our method requires $O(\frac{1}{\epsilon^2})$ samples to achieve calibration error $\epsilon$ instead of $O(\frac{B}{\epsilon^2})$ samples for histogram binning, where $B$ is the number of model outputs (Theorem~\ref{thm:final-calib}).
\pl{this is good overall, but I wonder if it's too much setup / notation from the intro, and it's strange that it's introduced for our method and not the general problem setting}

\begin{figure}
     \centering
     \begin{subfigure}[b]{0.32\textwidth}
         \centering
         \includegraphics[width=\textwidth]{platt_scaling}
         \caption{Platt scaling.}
         \label{fig:platt_scaling}
     \end{subfigure}
     \hfill
     \begin{subfigure}[b]{0.32\textwidth}
         \centering
         \includegraphics[width=\textwidth]{histogram_binning_with_deltas}
         \caption{Histogram binning.}
         \label{fig:hist_binning}
     \end{subfigure}
     \hfill
     \begin{subfigure}[b]{0.32\textwidth}
         \centering
         \includegraphics[width=\textwidth]{variance_reduced_binning_with_deltas}
         \caption{Variance-reduced binning.}
         \label{fig:var_red_binning}
     \end{subfigure}
        \caption{
        Visualization of the 3 recalibration approaches.
        The black crosses are the ground truth labels, and the red lines are the output of the recalibration methods.
        Platt Scaling (Figure~\ref{fig:platt_scaling}) fits a function to the recalibration data, but its calibration error is not measurable.
        Histogram binning (Figure~\ref{fig:hist_binning}) outputs the average label in each bin.
        Our variance-reduced binning (Figure~\ref{fig:var_red_binning}) fits a function $g \in \mathcal{G}$ to the recalibration data and then \emph{takes the average of the function values (the gray circles)} in each bin.
        The function values have lower variance than the labels, as visualized by the blue dotted lines, which is why our approach has lower variance. 
        }
        \label{fig:variance_reduced_illustration}
\end{figure}

Next, we turn to the question of efficiently measuring the calibration error of a model.
Prior work in machine learning~\cite{nguyen2015posterior, guo2017calibration, hendrycks2019anomaly, kuleshov2015calibrated, hendrycks2019pretraining} uses the plugin estimator for the calibration error (Definition~\ref{dfn:plugin-estimator}) where each term is empirically estimated from samples\pl{say in layman's terms}.
The sample complexity of the plugin estimator scales linearly with $B$.
We find that an alternative estimator introduced in the meteorological literature has sample complexity that scales with $\sqrt{B}$ (Theorem~\ref{thm:final-ours}).
We show that it achieves this by leveraging error cancellations across bins.

We run multi-class calibration experiments on CIFAR-10~\cite{krizhevsky2009learningmultiple} and ImageNet~\cite{deng2009imagenet}.
The objective is to minimize the mean-squared error, also known as the Brier score~\cite{brier1950verification}, subject to a calibration error budget.
We show that the variance-reduced calibrator achieves a lower mean-squared error for any given calibration constraint.
For example, we get a \emph{2x lower mean-squared error on CIFAR-10 if we want a calibration error $\leq 3\%$.}
We also show that the canceling estimator \tm{is th term  canceling estimator introduced before? }converges to the true calibration error much faster than the plugin estimator.

\tm{Sometimes I do a summary of contribution paragraph for the reviewers to easily see the contribution. doesn't add anything to the quality of the paper, but sometimes useful for paper acceptance .. (As a rushed reviewer sometimes I wanted the authors to have a simple para with a bit more technical term than abstract....)}

\section{Problem formulation}
\label{sec:formulation}

\newcommand{\lpce}[0]{\ensuremath{\ell_p\mbox{-CE}}}
\newcommand{\ltwoce}[0]{\ensuremath{\ell_2\mbox{-CE}}}
\newcommand{\squaredce}[0]{\ensuremath{\ell_2^2\mbox{-CE}}}
\newcommand{\topsquaredce}[0]{\ensuremath{\ell_2^2\mbox{-TCE}}}
\newcommand{\margsquaredce}[0]{\ensuremath{\ell_2^2\mbox{-MCE}}}

\subsection{Binary classification}

Let $\mathcal{X}$ be the input space and $\mathcal{Y}$ be the label space where $\mathcal{Y} = \{0, 1\}$ in the binary classification setting.
Let $X \in \mathcal{X}$ and $Y \in \mathcal{Y}$ be random variables denoting the input and label, given by an unknown joint distribution $P(X, Y)$.

Suppose we have a model $f : \mathcal{X} \to [0, 1]$ where the output of the model represents the model's confidence that the label is 1. $f$ may not be calibrated -- we define the calibration error, which examines the difference between the model's probability and the true probability given the model's output. If the calibration error is $0$ then the model is perfectly calibrated.

% \pl{in general, before introducing a formula, I always try to give the informal English description that conveys the core essence;
% here it'd be: something like 'calibration error, which examines the difference between the model's probability and true probability given the model's output'}
\begin{definition}[Calibration error]
For $p \geq 1$, the $\ell_p$ calibration error of $f : \mathcal{X} \to [0, 1]$ is given by:
\begin{align}
\lpce(f) = \Big(\expect\big[ (f(X) - \expect[Y \; | \; f(X)])^p \big] \Big)^{1/p}
\end{align}
\end{definition}
\pl{I like to use align for everything so equations be referencable}
\pl{$\ell_p\mbox{-CE}$ - is this standard? looks super awkward; can we do $\text{CE}_p$ or something? in any case, put it behind a macro}

The $\ell_2^2$ calibration error~\cite{nguyen2015posterior, hendrycks2019anomaly, kuleshov2015calibrated, hendrycks2019pretraining, murphy1973vector,degroot1983forecasters} is most commonly used but the $\ell_1$ and $\ell_{\infty}$ calibration errors are also used in the literature~\cite{guo2017calibration}.

Calibration alone is not sufficient: consider an image dataset containing $50\%$ dogs and $50\%$ cats.
If $f$ outputs $0.5$ on all inputs, $f$ is calibrated but not very useful.
We often also wish to minimize the mean-squared error -- also known as the Brier score -- as defined below.

\begin{definition}
The mean-squared error of $f : \mathcal{X} \to [0, 1]$ is given by $\mbox{MSE}(f) = \mathbb{E}[(f(X) - Y)^2]$.
\end{definition}

We often want to minimize the MSE subject to a calibration constraint. Of course, these are not orthogonal because $\mbox{MSE} = 0$ implies perfect calibration -- in fact the MSE is the sum of the $\ell_2^2$ calibration error and a `sharpness' term~\cite{murphy1973vector,degroot1983forecasters, kuleshov2015calibrated}.

\subsection{Multi-class classification}

While calibration in binary classification is well-studied,
it's less clear what to do for multi-class, where multiple definitions abound, differing in their strength. In the multi-class setting, $\mathcal{Y} = [K]$, where $[K] = \{1, \dots, K\}$ and $f : \mathcal{X} \to [0, 1]^k$ outputs a confidence measure for each class in $[K]$.

\begin{definition}[Top-label calibration]
The $\ell_2$ calibration error is given by:
\begin{align}
\topsquaredce(f) = \expect\Big[ \Big( \prob\big(Y = \argmax_{j \in [k]} M(X)_j \mid \max_{j \in [k]} M(X)_j\big) - \max_{j \in [k]} M(X)_j \Big)^2 \Big]
\end{align}
\end{definition}

We would often like the model to be calibrated on less likely predictions as well -- imagine that a medical diagnosis system says there is a $50\%$ chance a patient has a benign tumor, a $10\%$ chance she has an aggressive form of cancer, and a $40\%$ chance she has one of a long list of other conditions. We would like the model to be calibrated on all of these predictions so we define the marginal calibration error which examines, \emph{for each class}, the difference between the model's probability and the true probability of that class given the model's output.

\begin{definition}[Marginal calibration error]
\label{dfn:marginal-ce}
Let $P(j)$ denote the probability of class $j$. The $\ell_2^2$ marginal calibration error is:
\begin{align}
\margsquaredce(f) = \sum_{j = 1}^k P(j) \mathbb{E}\big[ (M(X)_j - \expect[Y = j \mid M(X)_j])^2 \big]
\end{align}
\end{definition}

Note that prior works~\cite{guo2017calibration, hendrycks2019anomaly, hendrycks2019pretraining} often claim to perform multi-class calibration but only measure top-label calibration.
% In the Appendix we discuss other notions of multi-class calibration.

\pl{is marginal calibration standard terminology? makes sense when compared with joint, but sounds a bit strange in relation to top-label}

For notational simplicity, our theory focuses on the binary classification setting. We can transform top-label calibration into a binary calibration problem -- the model outputs a probability corresponding to its top prediction, and the label represents whether the model gets it correct or not. Marginal calibration can be transformed into $K$ one-vs-all binary calibration problems where for each $k \in [K]$ the model outputs the probability associated with the $k$-th class, and the label represents whether the input belongs to the $k$-th class or not. We look at both top-label calibration and marginal calibration in our experiments.

\subsection{Recalibration}

Since most machine learning models do not output calibrated probabilities out of the box~\cite{guo2017calibration, zadrozny2001calibrated} recalibration methods take the output of an uncalibrated model, and transform it into a calibrated probability. That is, we are given a trained model $f: \mathcal{X} \to \mathcal{Z}$ and we wish to learn a recalibrator $g : \mathcal{Z} \to [0, 1]$ such that $g \circ f$ is well-calibrated.

% We say $M$ is pooled-calibrated (Kuleshov et al) if $J$ is sampled uniformly at random from $[k]$ and:
% \[ P\Big(Y = J | f_J(x) = p) \Big) \quad \forall p \]


\section{Is Platt scaling calibrated?}
\label{sec:challenges-measuring}

In this section, we show why methods like Platt scaling and temperature scaling are (i) less calibrated than reported and (ii) it is difficult to tell how miscalibrated they are. The issue is that it is very difficult \pl{we will show it's impossible?} to measure the calibration error of models that output a continuous range of values. We show, both theoretically and with experiments on CIFAR-10 and ImageNet, why the calibration error of such models is \emph{underestimated}. We defer proofs to Appendix~\ref{sec:appendix-platt-not-calibrated}.

% We begin by reviewing the definition of uncertainty calibration in the binary classification setting. Let $\mathcal{X} \subseteq \mathbb{R}^d$ and $\mathcal{Y} = \{0, 1\}$. Let $X \in \mathcal{X}$ and $Y \in \mathcal{Y}$ be random variables denoting the input and label, given by an unknown joint distribution $P(X, Y)$. Suppose we have a model $f : \mathcal{X} \to [0, 1]$ where the output of the model represents the model's confidence that the label is 1.

% \begin{definition}
% A model $f : \mathcal{X} \to [0, 1]$ is perfectly calibrated if $s = E[Y | f(X) = s]$ for all $s \in [0, 1]$.
% \end{definition}

% \begin{definition}
% For $p \geq 1$, the $\ell_p$ calibration error of a model $f : \mathcal{X} \to [0, 1]$ is given by:
% \[ \ell_p\mbox{-CE}(f) = \Big(E\big[ (f(X) - E[Y | f(X)])^p \big] \Big)^{1/p} \]
% \end{definition}

% The $\ell_2^2$, $\ell_1$ and $\ell_{\infty}$ calibration errors are popular choices. For all $p$, the $\ell_p$ calibration error can be tricky to estimate from samples, because $E[Y | f(X)]$ is difficult to estimate. In particular, for a model $f$ that outputs a continuous range of values between $[0, 1]$, we usually see any given $f(X)$ value exactly once.
% This makes estimating $E[Y | f(X)$ impossible without additional assumptions on the smoothness of $E[Y | f(X)]$.

For all $p$, the $\ell_p$ calibration error can be tricky to estimate from samples, because $\expect[Y \mid f(X)]$ is difficult to estimate. \pl{seems like we're being very repetitive; say: The key to estimating $\ell_p$ is estimating a conditional expectation based on a continuous value;
without smoothness assumptions, this is impossible}
The core problem for a model $f$ that outputs a continuous range of values between $[0, 1]$ is that we usually see any given $f(X)$ value exactly once. This is analogous to the difficulty of measuring the mutual information between two continuous signals~\cite{paninski2003entropy}.

To approximate the $\ell_p$ calibration error, prior work bins the output of $f$ into $B$ intervals.
The calibration error in each bin is estimated as the difference between the average value of $f(X)$ and $Y$ in that bin.
Note that the binning here is for evaluation only, whereas in histogram binning is used for the \pl{recalibration} method itself.
We formalize the notion of this binned calibration error below.

% \ak{Is the above text enough to motivate these definitions, or should I add more text?}
% \begin{definition}
% A binning scheme $\mathcal{B}$ is a set of $B$ disjoint intervals $I_1, \cdots, I_B$ that cover $[0, 1]$.
% \end{definition}

\begin{definition}
The binned version of $f$ outputs the average value of $f$ in each bin:
\begin{align}
f_{\mathcal{B}}(x) = E[f(X) \mid f(X) \in I_j] \quad\quad\quad \mbox{where }x \in I_j
\end{align} 
\end{definition}
\pl{hoping you could just reuse the binning definition from setup, so you just have to write $\hat f$, or better $\bar f$ or $f_\text{binned}$ or some macro}

Given $\bins{}$, the binned calibration error of $f$ is simply $\lpce(f_{\bins{}})$.
A simple example shows that using binning to estimate the $\ell_p$ calibration error can severely underestimate the true $\ell_p$ calibration error, even for $p=1$, the average calibration error.

\begin{restatable}{example}{continuousNotCalibrated}
\label{ex:continuous-not-calibrated}
For any binning scheme $\bins{}$, $p \in \mathbb{Z}^+$, and continuous bijective function $f : [0, 1] \to [0, 1]$, there exists a distribution $P$ over $X, Y$ s.t. $\lpce(f_{\bins{}}) = 0$ but $\lpce(f) \geq 0.49$.
Note that for all $f$, $0 \leq \lpce(f) \leq 1$.
\end{restatable}

The intuition is that in each interval $I_j$ in $\bins{}$, the model could underestimate the true probability $\expect[Y \mid f(X)]$ half the time, and overestimate the probability half the time. So if we average over the entire bin the model appears to be calibrated, even though it is very uncalibrated. The formal construction is in Appendix~\ref{sec:appendix-platt-not-calibrated}. This may seem contrived but we will show experimental evidence of this phenomena as well \pl{cut this sentence}.
% \begin{proof}
% The intuition is that $E[Y | f(X)]$ can be a function that oscillates a lot (like a sine wave with very high frequency), but for each interval $I_j$ we still have $\hat{f}(I_j) = \hat{Y}(I_j)$. Formal proof in Appendix.
% \end{proof}

Next, we show that given a function $f$, its binned version always has lower calibration error.
% and that \emph{finer} binning schemes give us a better lower bound.

% \begin{definition}
% Let $\mathcal{B}$ given by intervals $I_1, ..., I_m$ and $\mathcal{B}'$ given by intervals $I_1', ..., I_n'$ be binning schemes. We say $\mathcal{B}' \preceq \mathcal{B}$ if for all $1 \leq j \leq n$, there exists $1 \leq k \leq m$ s.t. $I_j' \subseteq I_k$. 
% \end{definition}

% \begin{example}
% A finer binning scheme partitions $[0, 1]$ into a finer set of intervals. If $\mathcal{B}_1 = \{ (0, 0.5), (0.5, 1.0)\}$ and $\mathcal{B}_2 = \{(0, 0.2), (0.2, 0.5), (0.5, 0.75), (0.75, 1.0)\}$, then we $\mathcal{B}_2$ is a finer binning scheme than $\mathcal{B}_1$. On the other hand, $\mathcal{B}_3 = \{(0, 0.2), (0.2, 0.6), (0.6, 1.0)\}$ is not a finer binning scheme than $\mathcal{B}_1$. 
% \end{example}

\begin{restatable}[Binning underestimates error]{proposition}{binningLowerBound}
\label{prop:bin_low_bound}
  Given \pl{English} $\bins{}$ and model $f : \mathcal{X} \to [0, 1]$. We have:
\[  \lpce(f_{\bins{}}) \leq \lpce(f) \]
\end{restatable}

The proof is by Jensen's inequality. The intuition for why this holds is that \pl{shorten: Intuitively,} averaging a model's prediction within a bin allows errors at different parts of the bin to cancel out with each other. 

\subsection{Experiments}

Our experiments on ImageNet and CIFAR-10 suggest that previous work reports numbers which are lower than the actual calibration error of their models. We cannot compute the actual calibration error, however recall that \pl{run-on sentence} binning lower bounds the calibration error of a model. If we use a `finer' set of bins then we get a tighter lower bound on the calibration error than reported in past work.

As in~\cite{guo2017calibration}, the model's objective was to output the top predicted class and a confidence score associated with the prediction. For ImageNet, we used a trained VGG16 model with an accuracy of 64.3\%. We split the validation set into 3 sets of size $(20000, 5000, 25000)$. We used the first set of data to recalibrate the model using Platt scaling, the second to select the binning scheme $\mathcal{B}$ so that each bin contains an equal number of points, and the third to measure the binned calibration error. We calculated $90\%$ confidence intervals for the binned calibration error using 1,000 bootstrap resamples and performed the same experiment with varying numbers of bins.
% from the TensorFlow Keras library

Figure~\ref{fig:imagenet_lower_bound} shows that as we increase the number of bins on ImageNet, the measured calibration error is higher and this is statistically significant. For example, if we use 15 bins as in~\cite{guo2017calibration}, we would think the $\ell_2$ calibration error is around 0.02 when the calibration error is at least twice as high. Figure~\ref{fig:cifar_10_lower_bound} shows similar findings for CIFAR-10, and in Appendix~\ref{sec:appendix_platt_experiments} we show that our findings hold even if we use $\ell_1$ calibration error and alternative binning strategies.

% To see if this phenomena \pl{restate the phenomena in case people started reading from here} occurs in practice, we ran experiments on CIFAR-10 and Imagenet \pl{capitalize N}, which show that \pl{here we have
% to be very clear about how to interpret our result;
% remember, we want to show that previous work has reported numbers which are lower than the actual calibration because they used binning;
% Then say, we can't compute the calibration error of the final thing, but remember that binning can lower improve calibration error,
% so the calibration error of $f$ must be at least with a finer bin (i.e., walk people through the argument starting with what you want to show
% } increasing the number of bins uncovers a higher model calibration error.
% The model's objective was to output the top predicted class and a confidence score associated with the prediction.
% For CIFAR-10, we used a trained VGG-net model \pl{say where it came from (can defer to appendix or experimental section)} with an accuracy of 93.1\%.
% We then split the test set into 3 chunks of size $(1000, 1000, 8000)$.
% We used the first chunk of data to recalibrate the model using Platt Scaling \pl{lowercase}, the second to select the binning scheme $\mathcal{B}$ \pl{how?}, and the third to measure the binned calibration error.
% We calculated $90\%$ confidence intervals for the binned calibration error using 1,000 Bootstrap \pl{lowercase b} resamples.
% We performed the same experiment with varying numbers of bins --
% the results are shown in Figure~\ref{fig:lower_bounds}.
% \pl{rewrite: Figure~\ref{fig:lower_bounds} shows that as we increase the number of bins, [punchline]}
% \pl{need to interpret and spoon feed the result: say that according to the binned method of reporting (cite), they would have said calibration error of $f$
% was X, but this experiment shows that it is at least Y}
% We repeated this experiment on ImageNet using a trained VGG-net model with top-1 accuracy of 64.3\%, splitting the validation set into 3 chunks of size $(20000, 5000, 25000)$.
% \pl{and?}
% In the Appendix we show that similar results hold for the $\ell_1$ calibration error as well.  \pl{what's the point?}

\begin{figure}
     \centering
     \begin{subfigure}[b]{0.4\textwidth}
         \centering
         \includegraphics[width=\textwidth]{l2_lower_bound_imagenet_plot}
         \caption{ImageNet.}
         \label{fig:imagenet_lower_bound}
     \end{subfigure}
     \hfill
     \begin{subfigure}[b]{0.4\textwidth}
         \centering
         \includegraphics[width=\textwidth]{l2_lower_bound_cifar_plot}
         \caption{CIFAR-10.}
         \label{fig:cifar_10_lower_bound}
     \end{subfigure}
        \caption{
        Binned $\ell_2$ calibration errors of a recalibrated VGG-net model on CIFAR-10 and ImageNet with $90\%$ confidence intervals. The binned calibration error increases as we increase the number of bins. This suggests that binning cannot be reliably used to measure the true calibration error.
        }
        \label{fig:lower_bounds}
\end{figure}


% We might also wish to compare the calibration of different candidate models.

% \footnote{One possibility is to modify the calibration metric, for example to only require our model to be calibrated for all possible intervals of width $\geq \epsilon$.}

% One possibility is to modify the calibration metric, for example to only require our model to be calibrated for all possible intervals of width $\geq \epsilon$. 
% % This makes it difficult to ascertain if a model has a desired calibration error, or which of two models is better calibrated.

% We see three potential ways to resolve this:
% \begin{enumerate}
% \item Add a smoothness constraint on $E[Y | f(X)$], for example assume $E[Y | f(X)]$ is $L$-Lipschitz. Smoothness assumptions are common when training a model, but it is unsatisfying to assume a value of $L$, which we do not know, when \emph{evaluating} a model.
% \item Explore alternative metrics for calibration. For example, perhaps we only require our model to be calibrated for all possible intervals of width $\geq \epsilon$. 
% \item Discretize the outputs of the final model, so that it only outputs a finite number of values.
% \end{enumerate}

% In this paper we explore (3), but (1) and (2) are good directions for future research.



\section{The variance-reduced calibrator}
\label{sec:calibrating_models}

In the previous section we saw that the problem with scaling methods is we cannot estimate their calibration error. The upside of scaling methods is that if the function family has at least one function that can achieve calibration error $\epsilon$, they require $O(1/\epsilon^2)$ samples to reach calibration error $\epsilon$, while histogram binning requires $O(B/\epsilon^2)$ samples where $B$ can be large. Can we get a method that is sample efficient to calibrate and one where it's possible to estimate the calibration error? To achieve this, we propose the variance-reduced calibrator (Figure~\ref{fig:var_red_binning}) where we first fit a scaling function, and then bin the outputs of the scaling function. Note that in previous work binning the outputs of a scaling function was used for evaluation whereas here it is used for the method itself.

\subsection{Variance-reduced recalibration algorithm}

We split the recalibration data $T$ of size $n$ into 3 sets: $T_1$, $T_2$, $T_3$. The variance-reduced recalibration algorithm, illustrated in Figure~\ref{fig:variance_reduced_illustration}, outputs $\hat{g_{\mathcal{B}}}$ such that $\hat{g_{\mathcal{B}}} \circ f$ has low calibration error:

\textbf{Step 1 (Function fitting):} The first step is to select $g = \argmin_{g \in \mathcal{G}} \sum_{(z, y) \in T_1} (y - g(z))^2$.

\textbf{Step 2 (Bin construction):} The second step is to construct a suitable binning scheme. We choose the bins so that an equal number of $g(z_i)$ in $T_2$ land in each bin $I_j$ for each $j \in \{1, \dots, B\}$, instead of equal width bins used in~\cite{guo2017calibration}--our choice of bins is essential for our bounds.

\textbf{Step 3 (Discretization):} The third step is to discretize $g$, by outputting the average $g$ value in each bin---these are the gray circles in Figure~\ref{fig:var_red_binning}. Let $\mu(S) = \frac{1}{|S|} \sum_{s \in S} s$ denote the mean of a set of values $S$.
Let $\hat{\mu}[j] = \mu(\{ g(z_i) \; | \; g(z_i) \in I_j \wedge (z_i, y_i) \in T_3 \})$ be the mean of the $g(z_i)$ values that landed in the $j$-th bin.
Recall that if $z \in I_j$, $\beta(z) = j$ is the interval z lands in.
Then we set $\hat{g_{\mathcal{B}}}(z) = \hat{\mu}[\beta(g(z))]$---that is we simply output the mean value in the bin that $g(z)$ falls in.

\pl{why is $g$ not hatted and $\hat{g_\mathcal{B}}$ hatted? make consistent}

\ak{one reason is that I use $g_{\mathcal{B}}$ below. This is the binned version of $g$, assuming infinite data for binning. $\hat{g_\mathcal{B}}$ is the empirically binned version of $g$. The hat refers to the empirical binning. What do you think?}

\subsection{Analysis}

We now show that the variance-reduced calibrator requires $O(B + 1/\epsilon^2)$ samples to calibrate, and in Section~\ref{sec:verifying_calibration} we show that we can efficiently measure its calibration error. For the main theorem, we make some standard regularity assumptions on $\mathcal{G}$ (Lipschitz, injectivity, bounded parameters) which we formalize in Appendix~\ref{sec:calibrating_models_appendix}. Our result is a generalization result---we show that if $\mathcal{G}$ contains some $g^*$ with low calibration error, then our method will quickly find $\hat{g}_{\mathcal{B}}$ with low calibration error:

\begin{restatable}[Calibration bound]{theorem}{finalCalib}
\label{thm:final-calib}
Assume regularity conditions on $\mathcal{G}$ (finite parameters, injectivity, Lipschitz-continuity, consistency, twice differentiability) Given $\delta \in (0, 1)$, there is a constant $c$ such that \emph{for all} $B, \epsilon$, with $n \geq c \Big(B\log{B} + \frac{\log{B}}{\epsilon^2}\Big)$ samples, the variance-reduced algorithm finds $\hat{g}_{\mathcal{B}}$ with $\squaredce(\hat{g}_{\mathcal{B}}) \leq 2\min_{g \in \mathcal{G}}\squaredce(g) + \epsilon^2$, with probability $\geq 1 - \delta$.
% Suppose that $\min_{g \in \mathcal{G}}\lsquared\mbox{-CE}(g) \leq \epsilon^2$.
%   With $n = \widetilde{O}(B + \frac{d'}{\epsilon^2})$ samples, where $\widetilde{O}$ hides log factors, the variance-reduced calibration algorithm finds $\hat{g}_{\mathcal{B}}$ with $\lsquared\mbox{-CE}(\hat{g}_{\mathcal{B}}) \leq 2 \epsilon^2$. Note that $d'$ is the number of parameters which is $1$ for Platt scaling.
\end{restatable}

We present a proof in Appendix~\ref{sec:calibrating_models_appendix} but give a sketch here. Step 1 of our algorithm is Platt scaling, which simply fits a function $g$ to the data---standard results in asymptotic statistics show that $g$ converges in $O(\frac{1}{\epsilon^2})$ samples.

Step 3, where we bin the outputs of $g$, is the main variance-reduction step. If we had infinite data, Proposition~\ref{prop:bin_low_bound} showed that the binned version $g_{\bins{}}$ has lower calibration error than $g$, so we would be done. However we do not have infinite data---the core of our proof is to show that the empirically binned $\hat{g_{\bins{}}}$ converges to $g_{\bins{}}$ in $O(B + \frac{1}{\epsilon^2})$ samples, instead of $O(B + \frac{B}{\epsilon^2})$ samples in histogram binning. The intuition is in Figure~\ref{fig:variance_reduced_illustration}---the $g(z_i)$ values in each bin (gray circles in Figure~\ref{fig:var_red_binning}) are in a narrower range than the $y_i$ values (black crosses in Figure~\ref{fig:hist_binning}) so when we take the average, we incur less estimation error. The perhaps surprising part is that we are estimating $B$ numbers with $\widetilde{O}(1/\epsilon^2)$ samples. In fact, there may be a small number of bins where the $g(z_i)$ values are not in a narrow range, but our proof still shows that the overall estimation error is small.

In Section~\ref{sec:challenges-measuring} we showed that current techniques cannot measure the calibration error of scaling methods, in contrast we show that we \emph{can} measure the calibration error of our calibrator. Recall that we chose our bins so that each bin had an equal proportion of points in the recalibration set. Lemma~\ref{lem:well-balanced} will show that this property approximately holds in the population as well. This will allow us to estimate the calibration error efficiently (Theorem~\ref{thm:final-ours}).

\begin{definition}[Well-balanced binning]
Given a binning scheme $\mathcal{B}$ of size $B$, and $\alpha \geq 1$. We say $\mathcal{B}$ is $\alpha$-well-balanced if for all $j$,
  \[ \frac{1}{\alpha B} \leq \prob(Z \in I_j) \leq \frac{\alpha}{B}\]
\end{definition}

\begin{restatable}{lemma}{wellBalanced}
\label{lem:well-balanced}
If $n \geq B\log{\frac{B}{\delta}}$, with probability at least $1 - \delta$, the binning scheme $\mathcal{B}$ we chose is 2-well-balanced.
\end{restatable}

While the way we choose bins is not novel~\cite{zadrozny2001calibrated}, we believe the guarantees around it are---not all binning schemes in the literature allow us to efficiently estimate the calibration error, for example the binning scheme in~\cite{guo2017calibration} does not. Our proof of Lemma~\ref{lem:well-balanced} is in Appendix~\ref{sec:calibrating_models_appendix}. The core challenge is that applying Chernoff bounds or a standard VC dimension argument gives us a loose bound and tells us we need $O(B^2\log{\frac{B}{\delta}})$ samples. We use a discretization argument to prove the result.

In Proposition~\ref{prop:mse-finite-binning} in Appendix~\ref{sec:calibrating_models_appendix}, we also show that if we use many bins, binning the outputs has little impact on model quality as measured by the mean-squared error.

\subsection{Experiments}

Our experiments on CIFAR-10 and ImageNet show that in the low-data regime, for example when we use $\leq 1000$ data points to recalibrate, our variance-reduced calibration method produces models with much lower calibration error than histogram binning. The model's objective was to output a confidence score associated with each class, where we calibrated each class separately as in~\cite{zadrozny2002transforming}, used $B$ bins per class and evaluated calibration using the marginal calibration error (Definition~\ref{dfn:marginal-ce}).

We describe our experimental protocol for CIFAR-10.
The CIFAR-10 validation set has 10,000 data points. We sampled, with replacement, a recalibration set of 1,000 points. We ran either the variance-reduced calibrator (we fit a sigmoid in the function fitting step) or histogram binning and measured the marginal calibration error on the entire set of 10K points.
% \footnote{This is equivalent to using the empirical distribution on the 10K validation points as the true data distribution, and comparing how these methods perform. We do this since we cannot measure the ground truth calibration error.}
We repeated this entire procedure 100 times and computed mean and 90\% confidence intervals, and we repeated this varying the number of bins $B$. Figure~\ref{fig:marginal_calibrator_comparison_cifar} shows that the variance-reduced calibrator produces models with lower calibration error, for example $35\%$ lower calibration error when we use 100 bins per class.

Using more bins allows a model to produce more fine-grained predictions, e.g.~\cite{brocker2012empirical} use $B = 51$ bins, which improves the quality of predictions which can be measured by the mean-squared error -- Figure~\ref{fig:cifar_calibrator_cmp_mse_ce} shows that our method achieves better mean-squared errors for any given calibration constraint.\footnote{By the decomposition of the mean-squared error, this means our method achieves better `sharpness' as well.} More concretely, the figure shows a scatter plot of the mean-squared error and $\lsquared{}$ calibration error for histogram binning and variance-reduced calibration when we vary the number of bins. For example, if we want our models to have an $\lsquared{}$ calibration error $\leq 0.0004$ (which translates to an $\ell_2$ `average' calibration error of 2\%) we get a $9\%$ better mean-squared error. In Appendix~\ref{sec:calibrating_models_appendix_experiments} we show that we can get nearly \emph{5x lower top-label calibration error on ImageNet}, and give further experiment details.
% We also run an ablation to show that if we have more data points then as the theory predicts histogram binning catches up to the variance-reduced calibrator, and that the way we construct bins is crucial, the result do not hold if we use the binning scheme in~\cite{guo2017calibration}.

\begin{figure}
  \centering
  \centering
     \begin{subfigure}[b]{0.54\textwidth}
         \centering
         \includegraphics[width=0.8\textwidth]{marginal_ces.png}
         \caption{Effect of number of bins on $\lsquared{}$ calibration error.}
         \label{fig:marginal_calibrator_comparison_cifar}
     \end{subfigure}
     \hfill
     \begin{subfigure}[b]{0.44\textwidth}
         \centering
         \includegraphics[width=\textwidth]{marginal_mse_vs_ces.png}
         \caption{Tradeoff between calibration and MSE.}
         \label{fig:cifar_calibrator_cmp_mse_ce}
     \end{subfigure}
  \caption{
  (\textbf{Left}) Recalibrating using 1,000 data points on CIFAR-10, our variance-reduced calibrator achieves lower $\lsquared{}$ calibration error than histogram binning, especially when the number of bins $B$ is large.
  (\textbf{Right}) For a fixed calibration error, our variance-reduced calibrator allows us to use more bins. This results in models with more predictive power which can be measured by the mean-squared error. Note the Y-axis range is $[0.4, 0.8]$ to zoom into the relevant region.
  }
  \label{fig:nan2}
\end{figure}


\section{Verifying calibration}
\label{sec:verifying_calibration}

If a model outputs values from some finite set $S$,\footnote{The model can choose the set $S$.} and the probability of outputting each value $s \in S$ is not too small, then we can estimate its calibration error. The plugin estimator requires samples proportional to the number of model outputs $|S|$. Instead, we introduce the \emph{cancelling} estimator that requires samples proportional to $\sqrt{|S|}$. We prove these finite sample guarantees (Theorem~\ref{thm:final-ours}), and show experimental evidence that our estimators approximate the calibration error better. We show, experimentally, that using a better estimator allows us to pick out better models.

The plugin estimator directly estimates each term in the $\ell_2^2$ calibration error from samples. Suppose we wish to measure the calibration error of a model $f : \mathcal{X} \to S$ where $S \subseteq [0, 1]$. Order the elements in $S$: $s_1 < s_2 < \dots < s_B$, where $B = |S|$ denotes the number of model outputs. Suppose we get an evaluation set $T_n = \{(x_1, y_1), \dots, (x_n, y_n)\}$ sampled i.i.d. from $P(X, Y)$.

\begin{definition}[Plugin estimator]
Let $R_i = \{ y_j \; | \; (x_j, y_j) \in T_n\wedge f(x_j) = s_i \}$ denote the label values where the model outputs $s_i$.

We estimate $E[Y | f(X) = s_i]$ as:
\[ \hat{y_i} = \sum_{y \in R_i} \frac{y}{|R_i|} \] 
We estimate $P(f(X) = s_i)$ by:
\[ \hat{p_i} = \frac{|R_i|}{n} \]
Then the plugin estimator for the $\ell_2^2$ error is:
\[ \hat{E}_{\mbox{pl}}^2 = \sum_{i=1}^b \hat{p_i} (s_i - \hat{y_i})^2 \]
\end{definition}

We now define our improved estimator, which debiases the plugin estimator.

\begin{definition}[Cancelling estimator]
The cancelling estimator for the $\ell_2^2$ error is:
\[ \hat{E}^2 = \sum_{i=1}^b \hat{p_i} \Big[ (s_i - \hat{y_i})^2 - \frac{\hat{y_i}(1 - \hat{y_i})}{\hat{p_i}n-1} \Big] \]
\end{definition}

An important condition for verifying calibration is $p_i$ cannot be too small.

\begin{definition}
\label{dfn:low-bal}
We say a model is $k$-lower-balanced if $p(X = s_i) \geq \frac{1}{kB}$ for all $i$.
\end{definition}
 
We now analyze the two estimators. The main results are that to check if the $\ell_2^2$ calibration error of a model is $\leq \epsilon^2$, the plugin estimator requires $\Theta(\frac{b}{\epsilon^2})$ samples while our estimator requires $\theta(\frac{\sqrt{b}}{\epsilon^2})$ samples. See Theorem~\ref{thm:final-plugin} and Theorem~\ref{thm:final-ours} for these main results.

\textbf{We use the following notation simplification} to simplify the theorem statements and proofs:
\[ p_i = P(f(X) = s_i) \]
\[ y_i^* = \mathbb{E}[Y \; | \; f(X) = s_i] \]
\[ e_i = (s_i - y_i^*) \]

Then, if we let ${E^*}^2$ denote the actual $\ell_2^2$ calibration error, we have:
\[ {E^*}^2 = \sum_{i=1}^b p_i e_i^2 \]
\subsection{Analysis of plugin estimator}
\begin{lemma}
\label{lemma:c_n_lemma}
For some $k > 1$, suppose $p_i \geq \frac{1}{kb}$ for all $i$. Then if $n \geq 3kb \log{\frac{2b}{\delta}}$, we have $P(\forall i . \; |p_i - \hat{p_i}| \leq c(n) p_i) \geq 1 - \delta$ where
\[ c(n) = \sqrt{\frac{3kb \log{\frac{2b}{\delta}}}{n}} \]
\end{lemma}

% \begin{lemma}
% The plugin estimator satisfies the following decomposition:
% \[ \hat{E}^2 = \underbrace{\sum_{i=1}^b \hat{p_i}e_i^2}_{(1)}  - \underbrace{2\sum_{i=1}^b \hat{p_i}e_i(\hat{y_i} - y_i^*)}_{(2)} + \underbrace{\sum_{i=1}^b \hat{p_i}(\hat{y_i} - y_i^*)^2}_{(3)} \]
% \end{lemma}

\begin{theorem}
\label{thm:plugin-bound}
For some $k > 1$, suppose $p_i \geq \frac{1}{kb}$ for all $i$. Then for the plugin estimator, if $n \geq 3kb \log{\frac{2b}{\delta}}$, with probability at least $1 - 3\delta$:
\[ | \hat{E_{\mbox{pl}}}^2 - {E^*}^2 | \leq c(n){E^*}^2 + \sqrt{\frac{2(1+c(n)){E^*}^2}{n} \log{\frac{2}{\delta}}} + \frac{b}{2n} \log{\frac{2b}{\delta}} \]
Where $c(n)$ is defined in Lemma~\ref{lemma:c_n_lemma}
\end{theorem}

% Theorem~\ref{thm:plugin-bound} gives a sample complexity bound for the estimation error. In many cases we are interested in checking if our model has calibration error $\leq \epsilon$. In other words, we are given $\epsilon$. If the calibration error is $> \epsilon$, then with probability at least $1 - \delta$ we should output that it is not calibrated. $1-\delta$ is the significance of the test. If the calibration error is $< r\epsilon$, then with probability at least $1 - \delta'$, we should output that the model is calibrated. $1 - \delta'$ is the power at effect size $r < 1$. Typically, we will choose $\delta = \delta'$.
Typically, we are interested in checking if our model has calibration error $\leq \epsilon$. In other words, we are given $\epsilon, \delta > 0$ and effect size $r < 1$. If the calibration error is $> \epsilon$, then with probability at least $1 - \delta$ we should output that it is not calibrated. If the calibration error is $< r\epsilon$, then with probability at least $1 - \delta$, we should output that the model is calibrated. Another way of thinking about this is that we want to estimate the calibration error within a constant multiplicative factor.

\begin{theorem}
\label{thm:final-plugin}
Using the plugin estimator $\hat{E}_{\mbox{pl}}$, if $n = \Theta(kb + \frac{b}{\epsilon^2})$ ignoring $\log$ factors, we can check if ${E^*}^2 \leq \epsilon^2$ with failure probability $\delta$, and constant effect size $r$. 
\end{theorem}

% \begin{corollary}
% \label{cor:final-plugin}
% Using the plugin estimator $\hat{E}_{\mbox{pl}}$, if $n = \Theta(kb + \frac{b}{\epsilon^2})$ ignoring $\log$ factors, we can check if $|{E^*} | \leq \epsilon$ with significance and power $\delta$, and constant effect size $r$. 
% \end{corollary}

\subsection{Analysis of our estimator}

\begin{theorem}
\label{thm:our-bound}
For some $k > 1$, suppose $p_i \geq \frac{1}{kb}$ for all $i$. Then for the plugin estimator, if $n \geq 3kb \log{\frac{2b}{\delta}}$, with probability at least $1 - 4\delta$:
\[ | \hat{E}^2 - {E^*}^2 | \leq c(n){E^*}^2 + \sqrt{\frac{2(1+c(n)){E^*}^2}{n} \log{\frac{2}{\delta}}} + \frac{1}{n} + \frac{\sqrt{b}}{n}\log{\frac{4n}{\delta}} \log{\frac{2}{\delta}}\]
Where $c(n)$ is defined in Lemma~\ref{lemma:c_n_lemma}
\end{theorem}

\begin{theorem}
\label{thm:final-ours}
Using our estimator $\hat{E}$, if $n = \Theta(kb + \frac{\sqrt{b}}{\epsilon^2})$ ignoring $\log$ factors, we can check if ${E^*}^2  \leq \epsilon^2$ with failure probability $\delta$, and constant effect size $r$. 
\end{theorem}

% This means that our estimator has a substantially better dependency on the number of outputs of the model.

% \begin{corollary}
% \label{cor:final-ours}
% Using our estimator $\hat{E}$, if $n = \Theta(kb + \frac{\sqrt{b}}{\epsilon^2})$ ignoring $\log$ factors, we can check if $|{E^*} | \leq \epsilon$ with significance and power $\delta$, and constant effect size $r$. 
% \end{corollary}

\section{Experiments}

\subsection{Estimator accuracy}

We ran experiments on CIFAR-10 and Imagenet to compare the practical performance of our estimator and the plugin estimator. Suppose we have a model with $B$ outputs that has true $\ell_2^2$ calibration error ${E^*}^2$. Suppose we draw $n$ samples, and obtain estimates $\hat{E}^2$ and $\hat{E}_{pl}^2$ from our estimator and the plugin estimator respectively, where $\hat{E}^2$ and $\hat{E}_{pl}^2$ are random variables. We want to know which of these estimates tends to be closer to the true value ${E^*}^2$. One way is to compare the mean squared error of the estimates: $\mathbb{E}\big[ (\hat{E}^2 - {E^*}^2)^2 \big]$  and $\mathbb{E}\big[ (\hat{E}_{pl}^2 - {E^*}^2)^2 \big]$, for varying values of $B$ and $n$, which we show in Figure~\ref{fig:mse_estimators_bins}. We might also wish to compare histograms of $|\hat{E}^2 - {E^*}^2|$ and $|\hat{E}_{pl}^2 - {E^*}^2|$ for some fixed sample size $n$, which we show in Figure~\ref{fig:histograms_estimators_bins}.

\begin{figure}
  \centering
  \centering
     \begin{subfigure}[b]{0.45\textwidth}
         \centering
         \includegraphics[width=\textwidth]{mse_estimators_10_bins.png}
         \caption{$B$ = 10 bins}
         \label{fig:y equals x}
     \end{subfigure}
     \hfill
     \begin{subfigure}[b]{0.45\textwidth}
         \centering
         \includegraphics[width=\textwidth]{mse_estimator_100_bins.png}
         \caption{$B = 100$ bins}
         \label{fig:three sin x}
     \end{subfigure}
  \caption{Mean-squared errors of plugin and our estimators on a recalibrated VGG-net model on CIFAR-10 with $90\%$ confidence intervals (lower values better). Our estimator is closer to the ground truth -- the difference between the estimators is more pronounced when the number of bins is higher.}
  \label{fig:mse_estimators_bins}
\end{figure}

\begin{figure}
  \centering
  \centering
     \begin{subfigure}[b]{0.45\textwidth}
         \centering
         \includegraphics[width=\textwidth]{histogram_estimators_10_bins.png}
         \caption{$B$ = 10 bins}
     \end{subfigure}
     \hfill
     \begin{subfigure}[b]{0.45\textwidth}
         \centering
         \includegraphics[width=\textwidth]{histogram_estimates_100_bins.png}
         \caption{$B = 100$ bins}
     \end{subfigure}
  \caption{Histograms of the absolute value of the difference between estimated and ground truth $\ell_2^2$ calibration errors. For $B = 10$ bins, the results are mixed but we avoid very bad estimates. For $B=100$ our estimates are much closer to ground truth.}
  \label{fig:histograms_estimators_bins}
\end{figure}


We give an overview of our experimental setup here, with more details in the Appendix, and all code on Github. For CIFAR-10, we used a trained VGG-net model that we recalibrated using Platt scaling on a small held-out set with 1,000 data points. We use 2,000 data points to select $B$ bins to discretize the outputs of the model. The bins were selected so that each bin has an equal number of data points. We use the remaining data points (7,000 for CIFAR-10) for our experiments. 

Suppose we want to evaluate how good our estimates are when they are allowed to use $m$ samples. Ideally, we would draw $m$ samples from the true data distribution, and measure the difference between our estimate and the ground truth. This gives us a single data point, so we would need to do this $T$ times and compute the mean and confidence intervals. This requires an exhaustive amount of data, for example if $m = 5000$ and $T = 100$, we would need 500K data points just for evaluation.

Instead, we approximate the true data distribution with the empirical distribution on the 7,000 data points we have. Then, we measure all quantities with respect to the empirical distribution on the 7,000 data points. That is, we draw samples of size $m$ (with replacement) from the 7,000 data points and run the estimators on these samples. We compare these estimates with the ground truth, the calibration error measured on the empirical distribution. We repeat this $T$ times to get the mean and confidence intervals. This can be viewed as a middle ground between using simulated data, and the impractical ideal where we need 500K CIFAR-10 samples.


\subsection{Model Selection}

Often, the goal is to maximize the predictive quality of the model (for example, as measured by the Brier score), while ensuring that the calibration error is under some threshold $\epsilon$. In our experiment on CIFAR-10, we find that our estimator allows us to select models with a better Brier score subject to a given calibration constraint. The reason for this is that our estimator better bounds the calibration error, which allows us to use more bins while still bounding the calibration error below the desired range (Figure~\ref{fig:mse_vs_ce}. 


We used a trained VGG-net model that we recalibrated using Platt scaling on a small held-out set with 1,000 data points. We use 2,000 data points to select $B$ bins to discretize the outputs of the model, where $B$ ranged from 10 to 100. The bins were selected so that each bin has an equal number of data points. We use the remaining data points (7000 in the case of CIFAR-10) to measure the calibration error. We use bootstrap to estimate the standard error of the estimate, and we report a 95\% upper confidence bound on the estimate. This represents our estimated upper bound on the calibration error. We also report the Brier score. We compute these values for $B = 10, 15, \cdots, 100$ and show the Pareto curve. \footnote{That is, if a point both has worse calibration error and Brier score than another point, we exclude it.} For a given upper bound on the calibration error, our estimator enables us to use more bins which leads to a lower Brier score.
\begin{figure}
  \centering
  \includegraphics[scale=0.75]{mse_vs_verified_error_plugin_vs_ours.png}
  \caption{Brier scores and upper bounds on the calibration error computed by our estimator and the plugin estimator, when we vary the number of bins $B$. For a given calibration error, our estimator enables us to choose models with a better Brier score. For example, if we want a model with $\ell_2$ calibration error less than 0.015, we can use 100 bins, while the plugin estimator only lets us use 15 bins and incurs a 13\% higher Brier score.}
  \label{fig:mse_vs_ce}
\end{figure}

\bibliographystyle{unsrt}
\bibliography{all}

\appendix

\input{appendix}
% , the $\ell_p$ calibration error is difficult to measure. Previous work 

% The $\ell_p$ calibration error is given by:

% \begin{definition}
% Let $f^* : \mathcal{X} \to [0, 1]$ be the sharpest calibrator, defined as $f^*(x) = \mathbb{E}_P[Y | X = x]$. 
% \end{definition}

% \begin{definition}
% Given $f : \mathcal{X} \to [0, 1]$, we define $\omega(f) : \mathcal{X} \to [0, 1]$ by $\omega(f)(x) = \mathbb{E}_P[Y | f(X) = x]$. That is, $\omega(f)$ is the calibrated version of $f$. If $f$ is injective, then $\omega(f) = f^*$, but otherwise $\omega(f) \neq f^*$. 
% \end{definition}

% We focus on the $\ell_2$ calibration error metric, which is the most commonly used metric.

% \begin{definition}
% Given $f, g : \mathcal{X} \to [0, 1]$, we define $||f - g||_2^2 = \mathbb{E}_P[(f(X) - g(X))^2]$ and $||f- g||_2 = \sqrt{||f - g||_2^2}$. We then define the calibration error as $\ell_2\mbox{-CE}(f) = ||f - \omega(f))||_2$.
% \end{definition}


\end{document}