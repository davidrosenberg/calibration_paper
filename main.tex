\documentclass{article}
\pdfoutput=1

% if you need to pass options to natbib, use, e.g.:
%     \PassOptionsToPackage{numbers, compress}{natbib}
% before loading neurips_2019

% ready for submission
\usepackage[final, nonatbib]{neurips_2019}
\usepackage{amsfonts}
\usepackage{amsmath}
\usepackage{amsthm}
\usepackage{amssymb}
\usepackage{graphicx}
\usepackage{caption}
\usepackage{subcaption}
\usepackage{xcolor}
\usepackage[utf8]{inputenc}
\usepackage{thmtools, thm-restate}
% \usepackage{xpatch}
% \usepackage{apptools}

% \makeatletter
% \xpatchcmd{\thmt@restatable}% Edit \thmt@restatable
% {\csname #2\@xa\endcsname\ifx\@nx#1\@nx\else[{#1}]\fi}% Replace this code
% {\IfAppendix{\csname #2\@xa\endcsname}{\csname #2\@xa\endcsname\ifx\@nx{Restatement}\@nx\else[{Restatement}]\fi}}% with this code
% {}{} % execute code for success/failure instances
% \makeatother

\newtheorem{theorem}{Theorem}[section]
\newtheorem{lemma}[theorem]{Lemma}
\newtheorem{definition}[theorem]{Definition}
\newtheorem{example}[theorem]{Example}
\newtheorem{proposition}[theorem]{Proposition}
\newtheorem{corollary}[theorem]{Corollary}

\def\shownotes{1}  \ifnum\shownotes=0
\newcommand{\authnote}[2]{$\ll$\textsf{\footnotesize #1 notes: #2}$\gg$}
\else
\newcommand{\authnote}[2]{}
\fi
\newcommand{\tnote}[1]{{\color{blue}\authnote{Tengyu}{#1}}}



\newcommand{\pl}[1]{}
\newcommand{\tm}[1]{}
\newcommand{\ak}[1]{}

% \newcommand{\pl}[1]{\textcolor{red}{[PL: #1]}}
% \newcommand{\tm}[1]{\textcolor{blue}{[TM: #1]}}
% \newcommand{\ak}[1]{\textcolor{brown}{[AK: #1]}}

\newcommand{\expect}[0]{\ensuremath{\mathbb{E}}}
\newcommand{\prob}[0]{\ensuremath{\mathbb{P}}}
\newcommand{\ourcal}[0]{the scaling-binning calibrator}
\newcommand{\Ourcal}[0]{The scaling-binning calibrator}

\graphicspath{ {images/} }

% to compile a preprint version, e.g., for submission to arXiv, add add the
% [preprint] option:
%     \usepackage[preprint]{neurips_2019}

% to compile a camera-ready version, add the [final] option, e.g.:
    %  \usepackage[final]{neurips_2019}

% to avoid loading the natbib package, add option nonatbib:
%     \usepackage[nonatbib]{neurips_2019}

\usepackage[numbers]{natbib}
\usepackage[utf8]{inputenc} % allow utf-8 input
\usepackage[T1]{fontenc}    % use 8-bit T1 fonts
\usepackage{hyperref}       % hyperlinks
\usepackage{url}            % simple URL typesetting
\usepackage{booktabs}       % professional-quality tables
\usepackage{amsfonts}       % blackboard math symbols
\usepackage{nicefrac}       % compact symbols for 1/2, etc.
\usepackage{microtype}      % microtypography

\newcommand{\argmax}{\mathop{\mbox{argmax}}}
\newcommand{\argmin}{\mathop{\mbox{argmin}}}
\newcommand{\argsup}{\mathop{\mbox{argsup}}}
\newcommand{\bins}{\mathcal{B}}

\title{Verified Uncertainty Calibration}

% The \author macro works with any number of authors. There are two commands
% used to separate the names and addresses of multiple authors: \And and \AND.
%
% Using \And between authors leaves it to LaTeX to determine where to break the
% lines. Using \AND forces a line break at that point. So, if LaTeX puts 3 of 4
% authors names on the first line, and the last on the second line, try using
% \AND instead of \And before the third author name.

% \setcounter{footnote}{0}
% \renewcommand*{\thefootnote}{\fnsymbol{footnote}}

\author{%
  Ananya Kumar, Percy Liang, Tengyu Ma \\
  Department of Computer Science\\
  Stanford University\\
  \texttt{\{ananya, pliang, tengyuma\}@cs.stanford.edu} \\
  % examples of more authors
  % \And
  % Percy Liang \\
  % Department of Computer Science\\
  % Stanford University\\
  % \texttt{pliang@cs.stanford.edu} \\
  % \And
  % Tengyu Ma \\
  % Department of Computer Science\\
  % Stanford University\\
  % \texttt{tengyuma@stanford.edu} \\
}

\begin{document}

\maketitle

\begin{abstract}
Applications such as weather forecasting and personalized medicine demand models that output calibrated probability estimates---those representative of the true likelihood of a prediction. Most models are not calibrated out of the box but are recalibrated by post-processing model outputs. We find in this work that popular recalibration methods like Platt scaling and temperature scaling, are (i) less calibrated than reported and (ii) current techniques cannot estimate how miscalibrated they are. An alternative method, histogram binning, has measurable calibration error but is sample inefficient---it requires $O(B/\epsilon^2)$ samples, compared to $O(1/\epsilon^2)$ for scaling methods, where $B$ is the number of distinct probabilities the model can output. To get the best of both worlds, we introduce \ourcal{}, which first fits a parametric function that acts like a baseline for variance reduction and then bins the function values to actually ensure calibration. This requires only $O(1/\epsilon^2 + B)$ samples. We then show that methods used to estimate calibration error are suboptimal---we prove that an alternative estimator introduced in the meteorological community requires fewer samples---samples proportional to $\sqrt{B}$ instead of $B$. We validate our approach with multiclass calibration experiments on CIFAR-10 and ImageNet, where we obtain a 35\% lower calibration error than histogram binning and, unlike scaling methods, guarantees on true calibration.

% Applications such as weather forecasting and personalized medicine demand models that output calibrated probability estimates---those representative of the true likelihood of a prediction.
% Scaling recalibration methods that fit a parametric model are great, because they require $O(1/\epsilon^2)$ samples to achieve calibration error $\epsilon$, but we find in this work that they are (i) less calibrated than reported and (ii) current techniques cannot estimate how miscalibrated they are.
% An alternative method, histogram binning, is sample inefficient -- it requires $O(B/\epsilon^2)$ samples where $B$ is the number of distinct probabilities the model can output.
% To solve these problems, we introduce the variance-reduced calibrator, which first fits a parametric function that acts like a baseline for variance reduction and then bins the function values to actually ensure calibration.
% This requires only $O(1/\epsilon^2 + B)$ samples.\tm{changed the order of two summands}
% We also show that the methods used in the machine learning literature to estimate calibration error are suboptimal -- we prove that an alternative estimator introduced in the meteorological community requires fewer samples -- samples proportional to $\sqrt{B}$ instead of $B$.
% We validate our approach with multiclass calibration experiments on CIFAR-10 and ImageNet, where we get a 2x lower calibration error than histogram binning and guarantees on true calibration unlike scaling methods.
% In this paper, we first show that previous calibration methods like Platt scaling and temperature scaling are (i) less calibrated than reported and (ii) we do not know how miscalibrated they are, because prior work only gives lower bounds for their calibration error \pl{reframe this: previous work (possibly some of our reviewers) did not think they were giving lower bounds; should say instead that methods used by previous work to assess calibration underestimate the true calibration error}.
%   Other methods \pl{what other methods?} like histogram binning are sample inefficient -- they require $O(\frac{B}{\epsilon^2})$ samples to reach calibration error $\epsilon$, where $B$ is the number of model outputs \pl{be clear: which model? this looks like you're talking about number of classes; which is coincidentally what you want, I guess!}. To solve these problems, we introduce the variance-reduced calibrator.
%   The variance-reduced calibrator first fits a parametric function -- which acts like a baseline for variance reduction -- and then bins the function values to actually ensure calibration. This requires only $O(B + \frac{1}{\epsilon^2})$ samples. \pl{we can join some sentences to make it flow better}
%   \pl{compress: We also introduce a new estimator for verifying...}
%   We then turn to the question of how to verify calibration. We introduce a new estimator for the calibration error that requires fewer samples -- samples proportional to $\sqrt{B}$ instead of $B$. We prove finite sample guarantees for all our results, and validate our theory with multiclass calibration experiments on CIFAR-10 and ImageNet, where we get a 2x lower calibration error than histogram binning. \pl{and guarantees on true calibration unlike scaling methods}
\end{abstract}

%\pl{Writing tip: find ways to group statements into single sentences so that you get some hierarchical structure: each sentence should express a set of related ideas, and important ideas are at the beginning of sentences}

\section{Introduction}

% Applications like personalized medicine, meteorological forecasting, and natural language processing \pl{NLP is so broad...anything could go here; either motivate based on fields that already use calibration (good to stand on shoulders of giants) or say something substantive about why calibration is needed (danger of being too preachy (danger of being too preachy))} require that classifiers provide calibrated confidence measures in addition to their predictions~\cite{jiang2012calibrating, brocker2009decomposition, nguyen2015posterior}.
% In other words, the probability that a system outputs for an event should reflect the true frequency of that event: if an automated diagnosis system says 1,000 patient have cancer with probability 0.1, approximately 100 of them should indeed have cancer.
The probability that a system outputs for an event should reflect the true frequency of that event: if an automated diagnosis system says 1,000 patients have cancer with probability 0.1, approximately 100 of them should indeed have cancer.
In this case we say the model is uncertainty calibrated. The importance of this notion of calibration has been emphasized in personalized medicine~\cite{jiang2012calibrating}, meteorological forecasting~\cite{brocker2009decomposition} and natural language processing applications~\cite{nguyen2015posterior}.
As most machine learning models, including neural networks, do not output calibrated probabilities out of the box~\cite{guo2017calibration, zadrozny2001calibrated} recalibration methods take the output of an uncalibrated model, and transform it into a calibrated probability.
\emph{Scaling} approaches for recalibration include Platt scaling~\cite{platt1999probabilistic}, isotonic regression~\cite{zadrozny2002transforming}, and temperature scaling~\cite{guo2017calibration}. These methods are widely used, but do they actually produce calibrated probabilities?
% , and speech recognition~\cite{dong2011calibration}
% These methods use additional recalibration data to fit a simple function on top of the original model outputs.

% Important, check that it's not too similar to "On calibration of modern neural networks" paper.
% In many applications, classification models must not only be accurate, but should indicate when they may be incorrect. For example, in automated healthcare applications, control should be passed on to human doctors when the confidence of a disease diagnosis prediction is low. Specifically, classifiers should provide a \emph{calibrated} confidence measure in addition to its prediction. In other words, the probability that a system outputs for an event should reflect the true frequency of that event: of the times that a system says a patient has cancer with probability 0.3, 30\% of the time, the patient should indeed have cancer.
% Typically, complex models like neural networks do not output calibrated probabilities.
% Instead, recalibration methods take the output of an uncalibrated model, and transform it into a calibrated probability.
% Popular recalibration approaches include Platt scaling and temperature scaling.

% Recent advances in machine learning have dramatically increased predictive accuracy. Machine learning methods are now entrusted with making decisions in applications ranging from weather forecasting to medical diagnosis \pl{seems a bit naive...}. In many of these settings, classification models must not only be accurate, but should indicate when they may be incorrect. For example, in automated healthcare applications, control should be passed on to human doctors when the confidence of a disease diagnosis prediction is low. Specifically, classifiers should provide a \emph{calibrated} confidence measure in addition to its prediction. In other words, the probability that a system outputs for an event should reflect the true frequency of that event: of the times that a system says that it will rain with probability 0.3,  30\% of the time, it should rain.
% \pl{need citations; is rain prediction really driven by "modern ML"?}\tm{It also seems unncessary to introduce the rain example --- it seems to be a good introduction to people who didn't know it before, but not something to write in the first para of a paper?}

%  \pl{I'd try to combine with first paragraph so that it's all background}
% \pl{introduce calibration error as a central concern first;
% methods try to calibrate, but do they actually achieve it?
% }

\emph{We discover that these methods are less calibrated than reported.} Estimating the calibration of models can be very challenging, \tm{maybe "the calibration of models" -> "the calibration of models with continuous outputs"?}so past work approximates a model's calibration error using a finite set of bins. We show that by using more bins, we can uncover a higher calibration error for models on CIFAR-10 and ImageNet (Figure~\ref{fig:lower_bounds}). This is a fundamental limitation with approaches that output a continuous range of probabilities -- we show that their true calibration error may never be measurable with a finite number of bins (Example~\ref{ex:continuous-not-calibrated}).

An alternative approach, histogram binning~\cite{zadrozny2001calibrated}, outputs probabilities from a finite set.
Histogram binning can produce a model that is calibrated, and unlike scaling methods we can measure its calibration error, but it can be sample inefficient.
In particular, the number of samples required to calibrate scales linearly with the number of model outputs $B$~\cite{naeini2014binary}, which can be large particularly in the multi-class setting where $B$ typically scales with the number of classes.
Recalibration sample efficiency is crucial -- we often want to recalibrate our models in the presence of domain shift~\cite{hendrycks2019anomaly} or recalibrate a model trained on simulated data, and may have access to only a small labeled dataset from the target domain.
\pl{do they actually do calibration there? if not, that's a weird cite}

To get the sample efficiency of Platt scaling and the verification guarantees of histogram binning, \emph{we propose the variance-reduced calibrator} (Figure~\ref{fig:var_red_binning}).
As with other recalibration methods, we begin with a recalibration dataset $\{(z_1, y_1), ..., (z_n, y_n)\}$, where $z_i$ represents the uncalibrated model output and $y_i$ the ground truth label.
In the binary classification setting, $z_i \in [0, 1]$ and $y_i \in \{0, 1\}$.
We fit a simple function $g \in \mathcal{G}$ to the recalibration dataset.
We then bin the input space so that an equal number of $g(z_i)$ land in each bin.
In each bin, we output the average of the $g(z_i)$ values in that bin -- these are the gray circles in Figure~\ref{fig:var_red_binning}.
Binning ensures that the output probabilities are from a finite set, so we can check if it is calibrated.
In contrast, histogram binning takes the average of the $y_i$ values in each bin (Figure~\ref{fig:hist_binning}).
The motivation behind our method is that the $g(z_i)$ values in each bin are in a narrower range than the $y_i$ values -- when we take the average we incur less of an estimation error.
If $\mathcal{G}$ is well chosen, our method requires $O(\frac{1}{\epsilon^2})$ samples to achieve calibration error $\epsilon$ instead of $O(\frac{B}{\epsilon^2})$ samples for histogram binning, where $B$ is the number of model outputs (Theorem~\ref{thm:final-calib}).
\pl{this is good overall, but I wonder if it's too much setup / notation from the intro, and it's strange that it's introduced for our method and not the general problem setting}

\begin{figure}
     \centering
     \begin{subfigure}[b]{0.32\textwidth}
         \centering
         \includegraphics[width=\textwidth]{platt_scaling}
         \caption{Platt scaling.}
         \label{fig:platt_scaling}
     \end{subfigure}
     \hfill
     \begin{subfigure}[b]{0.32\textwidth}
         \centering
         \includegraphics[width=\textwidth]{histogram_binning_with_deltas}
         \caption{Histogram binning.}
         \label{fig:hist_binning}
     \end{subfigure}
     \hfill
     \begin{subfigure}[b]{0.32\textwidth}
         \centering
         \includegraphics[width=\textwidth]{variance_reduced_binning_with_deltas}
         \caption{Variance-reduced binning.}
         \label{fig:var_red_binning}
     \end{subfigure}
        \caption{
        Visualization of the 3 recalibration approaches.
        The black crosses are the ground truth labels, and the red lines are the output of the recalibration methods.
        Platt Scaling (Figure~\ref{fig:platt_scaling}) fits a function to the recalibration data, but its calibration error is not measurable.
        Histogram binning (Figure~\ref{fig:hist_binning}) outputs the average label in each bin.
        Our variance-reduced binning (Figure~\ref{fig:var_red_binning}) fits a function $g \in \mathcal{G}$ to the recalibration data and then \emph{takes the average of the function values (the gray circles)} in each bin.
        The function values have lower variance than the labels, as visualized by the blue dotted lines, which is why our approach has lower variance. 
        }
        \label{fig:variance_reduced_illustration}
\end{figure}

Next, we turn to the question of efficiently measuring the calibration error of a model.
Prior work in machine learning~\cite{nguyen2015posterior, guo2017calibration, hendrycks2019anomaly, kuleshov2015calibrated, hendrycks2019pretraining} uses the plugin estimator for the calibration error (Definition~\ref{dfn:plugin-estimator}) where each term is empirically estimated from samples\pl{say in layman's terms}.
The sample complexity of the plugin estimator scales linearly with $B$.
We find that an alternative estimator introduced in the meteorological literature has sample complexity that scales with $\sqrt{B}$ (Theorem~\ref{thm:final-ours}).
We show that it achieves this by leveraging error cancellations across bins.

We run multi-class calibration experiments on CIFAR-10~\cite{krizhevsky2009learningmultiple} and ImageNet~\cite{deng2009imagenet}.
The objective is to minimize the mean-squared error, also known as the Brier score~\cite{brier1950verification}, subject to a calibration error budget.
We show that the variance-reduced calibrator achieves a lower mean-squared error for any given calibration constraint.
For example, we get a \emph{2x lower mean-squared error on CIFAR-10 if we want a calibration error $\leq 3\%$.}
We also show that the canceling estimator \tm{is th term  canceling estimator introduced before? }converges to the true calibration error much faster than the plugin estimator.

\tm{Sometimes I do a summary of contribution paragraph for the reviewers to easily see the contribution. doesn't add anything to the quality of the paper, but sometimes useful for paper acceptance .. (As a rushed reviewer sometimes I wanted the authors to have a simple para with a bit more technical term than abstract....)}

\section{Problem formulation}
\label{sec:formulation}

\newcommand{\lpce}[0]{\ensuremath{\ell_p\mbox{-CE}}}
\newcommand{\ltwoce}[0]{\ensuremath{\ell_2\mbox{-CE}}}
\newcommand{\squaredce}[0]{\ensuremath{\ell_2^2\mbox{-CE}}}
\newcommand{\topsquaredce}[0]{\ensuremath{\ell_2^2\mbox{-TCE}}}
\newcommand{\margsquaredce}[0]{\ensuremath{\ell_2^2\mbox{-MCE}}}

\subsection{Binary classification}

Let $\mathcal{X}$ be the input space and $\mathcal{Y}$ be the label space where $\mathcal{Y} = \{0, 1\}$ in the binary classification setting.
Let $X \in \mathcal{X}$ and $Y \in \mathcal{Y}$ be random variables denoting the input and label, given by an unknown joint distribution $P(X, Y)$.

Suppose we have a model $f : \mathcal{X} \to [0, 1]$ where the output of the model represents the model's confidence that the label is 1. $f$ may not be calibrated -- we define the calibration error, which examines the difference between the model's probability and the true probability given the model's output. If the calibration error is $0$ then the model is perfectly calibrated.

% \pl{in general, before introducing a formula, I always try to give the informal English description that conveys the core essence;
% here it'd be: something like 'calibration error, which examines the difference between the model's probability and true probability given the model's output'}
\begin{definition}[Calibration error]
For $p \geq 1$, the $\ell_p$ calibration error of $f : \mathcal{X} \to [0, 1]$ is given by:
\begin{align}
\lpce(f) = \Big(\expect\big[ (f(X) - \expect[Y \; | \; f(X)])^p \big] \Big)^{1/p}
\end{align}
\end{definition}
\pl{I like to use align for everything so equations be referencable}
\pl{$\ell_p\mbox{-CE}$ - is this standard? looks super awkward; can we do $\text{CE}_p$ or something? in any case, put it behind a macro}

The $\ell_2^2$ calibration error~\cite{nguyen2015posterior, hendrycks2019anomaly, kuleshov2015calibrated, hendrycks2019pretraining, murphy1973vector,degroot1983forecasters} is most commonly used but the $\ell_1$ and $\ell_{\infty}$ calibration errors are also used in the literature~\cite{guo2017calibration}.

Calibration alone is not sufficient: consider an image dataset containing $50\%$ dogs and $50\%$ cats.
If $f$ outputs $0.5$ on all inputs, $f$ is calibrated but not very useful.
We often also wish to minimize the mean-squared error -- also known as the Brier score -- as defined below.

\begin{definition}
The mean-squared error of $f : \mathcal{X} \to [0, 1]$ is given by $\mbox{MSE}(f) = \mathbb{E}[(f(X) - Y)^2]$.
\end{definition}

We often want to minimize the MSE subject to a calibration constraint. Of course, these are not orthogonal because $\mbox{MSE} = 0$ implies perfect calibration -- in fact the MSE is the sum of the $\ell_2^2$ calibration error and a `sharpness' term~\cite{murphy1973vector,degroot1983forecasters, kuleshov2015calibrated}.

\subsection{Multi-class classification}

While calibration in binary classification is well-studied,
it's less clear what to do for multi-class, where multiple definitions abound, differing in their strength. In the multi-class setting, $\mathcal{Y} = [K]$, where $[K] = \{1, \dots, K\}$ and $f : \mathcal{X} \to [0, 1]^k$ outputs a confidence measure for each class in $[K]$.

\begin{definition}[Top-label calibration]
The $\ell_2$ calibration error is given by:
\begin{align}
\topsquaredce(f) = \expect\Big[ \Big( \prob\big(Y = \argmax_{j \in [k]} M(X)_j \mid \max_{j \in [k]} M(X)_j\big) - \max_{j \in [k]} M(X)_j \Big)^2 \Big]
\end{align}
\end{definition}

We would often like the model to be calibrated on less likely predictions as well -- imagine that a medical diagnosis system says there is a $50\%$ chance a patient has a benign tumor, a $10\%$ chance she has an aggressive form of cancer, and a $40\%$ chance she has one of a long list of other conditions. We would like the model to be calibrated on all of these predictions so we define the marginal calibration error which examines, \emph{for each class}, the difference between the model's probability and the true probability of that class given the model's output.

\begin{definition}[Marginal calibration error]
\label{dfn:marginal-ce}
Let $P(j)$ denote the probability of class $j$. The $\ell_2^2$ marginal calibration error is:
\begin{align}
\margsquaredce(f) = \sum_{j = 1}^k P(j) \mathbb{E}\big[ (M(X)_j - \expect[Y = j \mid M(X)_j])^2 \big]
\end{align}
\end{definition}

Note that prior works~\cite{guo2017calibration, hendrycks2019anomaly, hendrycks2019pretraining} often claim to perform multi-class calibration but only measure top-label calibration.
% In the Appendix we discuss other notions of multi-class calibration.

\pl{is marginal calibration standard terminology? makes sense when compared with joint, but sounds a bit strange in relation to top-label}

For notational simplicity, our theory focuses on the binary classification setting. We can transform top-label calibration into a binary calibration problem -- the model outputs a probability corresponding to its top prediction, and the label represents whether the model gets it correct or not. Marginal calibration can be transformed into $K$ one-vs-all binary calibration problems where for each $k \in [K]$ the model outputs the probability associated with the $k$-th class, and the label represents whether the input belongs to the $k$-th class or not. We look at both top-label calibration and marginal calibration in our experiments.

\subsection{Recalibration}

Since most machine learning models do not output calibrated probabilities out of the box~\cite{guo2017calibration, zadrozny2001calibrated} recalibration methods take the output of an uncalibrated model, and transform it into a calibrated probability. That is, we are given a trained model $f: \mathcal{X} \to \mathcal{Z}$ and we wish to learn a recalibrator $g : \mathcal{Z} \to [0, 1]$ such that $g \circ f$ is well-calibrated.

% We say $M$ is pooled-calibrated (Kuleshov et al) if $J$ is sampled uniformly at random from $[k]$ and:
% \[ P\Big(Y = J | f_J(x) = p) \Big) \quad \forall p \]


\section{Is Platt scaling calibrated?}
\label{sec:challenges-measuring}

In this section, we show why methods like Platt scaling and temperature scaling are (i) less calibrated than reported and (ii) it is difficult to tell how miscalibrated they are. The issue is that it is very difficult \pl{we will show it's impossible?} to measure the calibration error of models that output a continuous range of values. We show, both theoretically and with experiments on CIFAR-10 and ImageNet, why the calibration error of such models is \emph{underestimated}. We defer proofs to Appendix~\ref{sec:appendix-platt-not-calibrated}.

% We begin by reviewing the definition of uncertainty calibration in the binary classification setting. Let $\mathcal{X} \subseteq \mathbb{R}^d$ and $\mathcal{Y} = \{0, 1\}$. Let $X \in \mathcal{X}$ and $Y \in \mathcal{Y}$ be random variables denoting the input and label, given by an unknown joint distribution $P(X, Y)$. Suppose we have a model $f : \mathcal{X} \to [0, 1]$ where the output of the model represents the model's confidence that the label is 1.

% \begin{definition}
% A model $f : \mathcal{X} \to [0, 1]$ is perfectly calibrated if $s = E[Y | f(X) = s]$ for all $s \in [0, 1]$.
% \end{definition}

% \begin{definition}
% For $p \geq 1$, the $\ell_p$ calibration error of a model $f : \mathcal{X} \to [0, 1]$ is given by:
% \[ \ell_p\mbox{-CE}(f) = \Big(E\big[ (f(X) - E[Y | f(X)])^p \big] \Big)^{1/p} \]
% \end{definition}

% The $\ell_2^2$, $\ell_1$ and $\ell_{\infty}$ calibration errors are popular choices. For all $p$, the $\ell_p$ calibration error can be tricky to estimate from samples, because $E[Y | f(X)]$ is difficult to estimate. In particular, for a model $f$ that outputs a continuous range of values between $[0, 1]$, we usually see any given $f(X)$ value exactly once.
% This makes estimating $E[Y | f(X)$ impossible without additional assumptions on the smoothness of $E[Y | f(X)]$.

For all $p$, the $\ell_p$ calibration error can be tricky to estimate from samples, because $\expect[Y \mid f(X)]$ is difficult to estimate. \pl{seems like we're being very repetitive; say: The key to estimating $\ell_p$ is estimating a conditional expectation based on a continuous value;
without smoothness assumptions, this is impossible}
The core problem for a model $f$ that outputs a continuous range of values between $[0, 1]$ is that we usually see any given $f(X)$ value exactly once. This is analogous to the difficulty of measuring the mutual information between two continuous signals~\cite{paninski2003entropy}.

To approximate the $\ell_p$ calibration error, prior work bins the output of $f$ into $B$ intervals.
The calibration error in each bin is estimated as the difference between the average value of $f(X)$ and $Y$ in that bin.
Note that the binning here is for evaluation only, whereas in histogram binning is used for the \pl{recalibration} method itself.
We formalize the notion of this binned calibration error below.

% \ak{Is the above text enough to motivate these definitions, or should I add more text?}
% \begin{definition}
% A binning scheme $\mathcal{B}$ is a set of $B$ disjoint intervals $I_1, \cdots, I_B$ that cover $[0, 1]$.
% \end{definition}

\begin{definition}
The binned version of $f$ outputs the average value of $f$ in each bin:
\begin{align}
f_{\mathcal{B}}(x) = E[f(X) \mid f(X) \in I_j] \quad\quad\quad \mbox{where }x \in I_j
\end{align} 
\end{definition}
\pl{hoping you could just reuse the binning definition from setup, so you just have to write $\hat f$, or better $\bar f$ or $f_\text{binned}$ or some macro}

Given $\bins{}$, the binned calibration error of $f$ is simply $\lpce(f_{\bins{}})$.
A simple example shows that using binning to estimate the $\ell_p$ calibration error can severely underestimate the true $\ell_p$ calibration error, even for $p=1$, the average calibration error.

\begin{restatable}{example}{continuousNotCalibrated}
\label{ex:continuous-not-calibrated}
For any binning scheme $\bins{}$, $p \in \mathbb{Z}^+$, and continuous bijective function $f : [0, 1] \to [0, 1]$, there exists a distribution $P$ over $X, Y$ s.t. $\lpce(f_{\bins{}}) = 0$ but $\lpce(f) \geq 0.49$.
Note that for all $f$, $0 \leq \lpce(f) \leq 1$.
\end{restatable}

The intuition is that in each interval $I_j$ in $\bins{}$, the model could underestimate the true probability $\expect[Y \mid f(X)]$ half the time, and overestimate the probability half the time. So if we average over the entire bin the model appears to be calibrated, even though it is very uncalibrated. The formal construction is in Appendix~\ref{sec:appendix-platt-not-calibrated}. This may seem contrived but we will show experimental evidence of this phenomena as well \pl{cut this sentence}.
% \begin{proof}
% The intuition is that $E[Y | f(X)]$ can be a function that oscillates a lot (like a sine wave with very high frequency), but for each interval $I_j$ we still have $\hat{f}(I_j) = \hat{Y}(I_j)$. Formal proof in Appendix.
% \end{proof}

Next, we show that given a function $f$, its binned version always has lower calibration error.
% and that \emph{finer} binning schemes give us a better lower bound.

% \begin{definition}
% Let $\mathcal{B}$ given by intervals $I_1, ..., I_m$ and $\mathcal{B}'$ given by intervals $I_1', ..., I_n'$ be binning schemes. We say $\mathcal{B}' \preceq \mathcal{B}$ if for all $1 \leq j \leq n$, there exists $1 \leq k \leq m$ s.t. $I_j' \subseteq I_k$. 
% \end{definition}

% \begin{example}
% A finer binning scheme partitions $[0, 1]$ into a finer set of intervals. If $\mathcal{B}_1 = \{ (0, 0.5), (0.5, 1.0)\}$ and $\mathcal{B}_2 = \{(0, 0.2), (0.2, 0.5), (0.5, 0.75), (0.75, 1.0)\}$, then we $\mathcal{B}_2$ is a finer binning scheme than $\mathcal{B}_1$. On the other hand, $\mathcal{B}_3 = \{(0, 0.2), (0.2, 0.6), (0.6, 1.0)\}$ is not a finer binning scheme than $\mathcal{B}_1$. 
% \end{example}

\begin{restatable}[Binning underestimates error]{proposition}{binningLowerBound}
\label{prop:bin_low_bound}
  Given \pl{English} $\bins{}$ and model $f : \mathcal{X} \to [0, 1]$. We have:
\[  \lpce(f_{\bins{}}) \leq \lpce(f) \]
\end{restatable}

The proof is by Jensen's inequality. The intuition for why this holds is that \pl{shorten: Intuitively,} averaging a model's prediction within a bin allows errors at different parts of the bin to cancel out with each other. 

\subsection{Experiments}

Our experiments on ImageNet and CIFAR-10 suggest that previous work reports numbers which are lower than the actual calibration error of their models. We cannot compute the actual calibration error, however recall that \pl{run-on sentence} binning lower bounds the calibration error of a model. If we use a `finer' set of bins then we get a tighter lower bound on the calibration error than reported in past work.

As in~\cite{guo2017calibration}, the model's objective was to output the top predicted class and a confidence score associated with the prediction. For ImageNet, we used a trained VGG16 model with an accuracy of 64.3\%. We split the validation set into 3 sets of size $(20000, 5000, 25000)$. We used the first set of data to recalibrate the model using Platt scaling, the second to select the binning scheme $\mathcal{B}$ so that each bin contains an equal number of points, and the third to measure the binned calibration error. We calculated $90\%$ confidence intervals for the binned calibration error using 1,000 bootstrap resamples and performed the same experiment with varying numbers of bins.
% from the TensorFlow Keras library

Figure~\ref{fig:imagenet_lower_bound} shows that as we increase the number of bins on ImageNet, the measured calibration error is higher and this is statistically significant. For example, if we use 15 bins as in~\cite{guo2017calibration}, we would think the $\ell_2$ calibration error is around 0.02 when the calibration error is at least twice as high. Figure~\ref{fig:cifar_10_lower_bound} shows similar findings for CIFAR-10, and in Appendix~\ref{sec:appendix_platt_experiments} we show that our findings hold even if we use $\ell_1$ calibration error and alternative binning strategies.

% To see if this phenomena \pl{restate the phenomena in case people started reading from here} occurs in practice, we ran experiments on CIFAR-10 and Imagenet \pl{capitalize N}, which show that \pl{here we have
% to be very clear about how to interpret our result;
% remember, we want to show that previous work has reported numbers which are lower than the actual calibration because they used binning;
% Then say, we can't compute the calibration error of the final thing, but remember that binning can lower improve calibration error,
% so the calibration error of $f$ must be at least with a finer bin (i.e., walk people through the argument starting with what you want to show
% } increasing the number of bins uncovers a higher model calibration error.
% The model's objective was to output the top predicted class and a confidence score associated with the prediction.
% For CIFAR-10, we used a trained VGG-net model \pl{say where it came from (can defer to appendix or experimental section)} with an accuracy of 93.1\%.
% We then split the test set into 3 chunks of size $(1000, 1000, 8000)$.
% We used the first chunk of data to recalibrate the model using Platt Scaling \pl{lowercase}, the second to select the binning scheme $\mathcal{B}$ \pl{how?}, and the third to measure the binned calibration error.
% We calculated $90\%$ confidence intervals for the binned calibration error using 1,000 Bootstrap \pl{lowercase b} resamples.
% We performed the same experiment with varying numbers of bins --
% the results are shown in Figure~\ref{fig:lower_bounds}.
% \pl{rewrite: Figure~\ref{fig:lower_bounds} shows that as we increase the number of bins, [punchline]}
% \pl{need to interpret and spoon feed the result: say that according to the binned method of reporting (cite), they would have said calibration error of $f$
% was X, but this experiment shows that it is at least Y}
% We repeated this experiment on ImageNet using a trained VGG-net model with top-1 accuracy of 64.3\%, splitting the validation set into 3 chunks of size $(20000, 5000, 25000)$.
% \pl{and?}
% In the Appendix we show that similar results hold for the $\ell_1$ calibration error as well.  \pl{what's the point?}

\begin{figure}
     \centering
     \begin{subfigure}[b]{0.4\textwidth}
         \centering
         \includegraphics[width=\textwidth]{l2_lower_bound_imagenet_plot}
         \caption{ImageNet.}
         \label{fig:imagenet_lower_bound}
     \end{subfigure}
     \hfill
     \begin{subfigure}[b]{0.4\textwidth}
         \centering
         \includegraphics[width=\textwidth]{l2_lower_bound_cifar_plot}
         \caption{CIFAR-10.}
         \label{fig:cifar_10_lower_bound}
     \end{subfigure}
        \caption{
        Binned $\ell_2$ calibration errors of a recalibrated VGG-net model on CIFAR-10 and ImageNet with $90\%$ confidence intervals. The binned calibration error increases as we increase the number of bins. This suggests that binning cannot be reliably used to measure the true calibration error.
        }
        \label{fig:lower_bounds}
\end{figure}


% We might also wish to compare the calibration of different candidate models.

% \footnote{One possibility is to modify the calibration metric, for example to only require our model to be calibrated for all possible intervals of width $\geq \epsilon$.}

% One possibility is to modify the calibration metric, for example to only require our model to be calibrated for all possible intervals of width $\geq \epsilon$. 
% % This makes it difficult to ascertain if a model has a desired calibration error, or which of two models is better calibrated.

% We see three potential ways to resolve this:
% \begin{enumerate}
% \item Add a smoothness constraint on $E[Y | f(X)$], for example assume $E[Y | f(X)]$ is $L$-Lipschitz. Smoothness assumptions are common when training a model, but it is unsatisfying to assume a value of $L$, which we do not know, when \emph{evaluating} a model.
% \item Explore alternative metrics for calibration. For example, perhaps we only require our model to be calibrated for all possible intervals of width $\geq \epsilon$. 
% \item Discretize the outputs of the final model, so that it only outputs a finite number of values.
% \end{enumerate}

% In this paper we explore (3), but (1) and (2) are good directions for future research.



\section{The variance-reduced calibrator}
\label{sec:calibrating_models}

In the previous section we saw that the problem with scaling methods is we cannot estimate their calibration error. The upside of scaling methods is that if the function family has at least one function that can achieve calibration error $\epsilon$, they require $O(1/\epsilon^2)$ samples to reach calibration error $\epsilon$, while histogram binning requires $O(B/\epsilon^2)$ samples where $B$ can be large. Can we get a method that is sample efficient to calibrate and one where it's possible to estimate the calibration error? To achieve this, we propose the variance-reduced calibrator (Figure~\ref{fig:var_red_binning}) where we first fit a scaling function, and then bin the outputs of the scaling function. Note that in previous work binning the outputs of a scaling function was used for evaluation whereas here it is used for the method itself.

\subsection{Variance-reduced recalibration algorithm}

We split the recalibration data $T$ of size $n$ into 3 sets: $T_1$, $T_2$, $T_3$. The variance-reduced recalibration algorithm, illustrated in Figure~\ref{fig:variance_reduced_illustration}, outputs $\hat{g_{\mathcal{B}}}$ such that $\hat{g_{\mathcal{B}}} \circ f$ has low calibration error:

\textbf{Step 1 (Function fitting):} The first step is to select $g = \argmin_{g \in \mathcal{G}} \sum_{(z, y) \in T_1} (y - g(z))^2$.

\textbf{Step 2 (Bin construction):} The second step is to construct a suitable binning scheme. We choose the bins so that an equal number of $g(z_i)$ in $T_2$ land in each bin $I_j$ for each $j \in \{1, \dots, B\}$, instead of equal width bins used in~\cite{guo2017calibration}--our choice of bins is essential for our bounds.

\textbf{Step 3 (Discretization):} The third step is to discretize $g$, by outputting the average $g$ value in each bin---these are the gray circles in Figure~\ref{fig:var_red_binning}. Let $\mu(S) = \frac{1}{|S|} \sum_{s \in S} s$ denote the mean of a set of values $S$.
Let $\hat{\mu}[j] = \mu(\{ g(z_i) \; | \; g(z_i) \in I_j \wedge (z_i, y_i) \in T_3 \})$ be the mean of the $g(z_i)$ values that landed in the $j$-th bin.
Recall that if $z \in I_j$, $\beta(z) = j$ is the interval z lands in.
Then we set $\hat{g_{\mathcal{B}}}(z) = \hat{\mu}[\beta(g(z))]$---that is we simply output the mean value in the bin that $g(z)$ falls in.

\pl{why is $g$ not hatted and $\hat{g_\mathcal{B}}$ hatted? make consistent}

\ak{one reason is that I use $g_{\mathcal{B}}$ below. This is the binned version of $g$, assuming infinite data for binning. $\hat{g_\mathcal{B}}$ is the empirically binned version of $g$. The hat refers to the empirical binning. What do you think?}

\subsection{Analysis}

We now show that the variance-reduced calibrator requires $O(B + 1/\epsilon^2)$ samples to calibrate, and in Section~\ref{sec:verifying_calibration} we show that we can efficiently measure its calibration error. For the main theorem, we make some standard regularity assumptions on $\mathcal{G}$ (Lipschitz, injectivity, bounded parameters) which we formalize in Appendix~\ref{sec:calibrating_models_appendix}. Our result is a generalization result---we show that if $\mathcal{G}$ contains some $g^*$ with low calibration error, then our method will quickly find $\hat{g}_{\mathcal{B}}$ with low calibration error:

\begin{restatable}[Calibration bound]{theorem}{finalCalib}
\label{thm:final-calib}
Assume regularity conditions on $\mathcal{G}$ (finite parameters, injectivity, Lipschitz-continuity, consistency, twice differentiability) Given $\delta \in (0, 1)$, there is a constant $c$ such that \emph{for all} $B, \epsilon$, with $n \geq c \Big(B\log{B} + \frac{\log{B}}{\epsilon^2}\Big)$ samples, the variance-reduced algorithm finds $\hat{g}_{\mathcal{B}}$ with $\squaredce(\hat{g}_{\mathcal{B}}) \leq 2\min_{g \in \mathcal{G}}\squaredce(g) + \epsilon^2$, with probability $\geq 1 - \delta$.
% Suppose that $\min_{g \in \mathcal{G}}\lsquared\mbox{-CE}(g) \leq \epsilon^2$.
%   With $n = \widetilde{O}(B + \frac{d'}{\epsilon^2})$ samples, where $\widetilde{O}$ hides log factors, the variance-reduced calibration algorithm finds $\hat{g}_{\mathcal{B}}$ with $\lsquared\mbox{-CE}(\hat{g}_{\mathcal{B}}) \leq 2 \epsilon^2$. Note that $d'$ is the number of parameters which is $1$ for Platt scaling.
\end{restatable}

We present a proof in Appendix~\ref{sec:calibrating_models_appendix} but give a sketch here. Step 1 of our algorithm is Platt scaling, which simply fits a function $g$ to the data---standard results in asymptotic statistics show that $g$ converges in $O(\frac{1}{\epsilon^2})$ samples.

Step 3, where we bin the outputs of $g$, is the main variance-reduction step. If we had infinite data, Proposition~\ref{prop:bin_low_bound} showed that the binned version $g_{\bins{}}$ has lower calibration error than $g$, so we would be done. However we do not have infinite data---the core of our proof is to show that the empirically binned $\hat{g_{\bins{}}}$ converges to $g_{\bins{}}$ in $O(B + \frac{1}{\epsilon^2})$ samples, instead of $O(B + \frac{B}{\epsilon^2})$ samples in histogram binning. The intuition is in Figure~\ref{fig:variance_reduced_illustration}---the $g(z_i)$ values in each bin (gray circles in Figure~\ref{fig:var_red_binning}) are in a narrower range than the $y_i$ values (black crosses in Figure~\ref{fig:hist_binning}) so when we take the average, we incur less estimation error. The perhaps surprising part is that we are estimating $B$ numbers with $\widetilde{O}(1/\epsilon^2)$ samples. In fact, there may be a small number of bins where the $g(z_i)$ values are not in a narrow range, but our proof still shows that the overall estimation error is small.

In Section~\ref{sec:challenges-measuring} we showed that current techniques cannot measure the calibration error of scaling methods, in contrast we show that we \emph{can} measure the calibration error of our calibrator. Recall that we chose our bins so that each bin had an equal proportion of points in the recalibration set. Lemma~\ref{lem:well-balanced} will show that this property approximately holds in the population as well. This will allow us to estimate the calibration error efficiently (Theorem~\ref{thm:final-ours}).

\begin{definition}[Well-balanced binning]
Given a binning scheme $\mathcal{B}$ of size $B$, and $\alpha \geq 1$. We say $\mathcal{B}$ is $\alpha$-well-balanced if for all $j$,
  \[ \frac{1}{\alpha B} \leq \prob(Z \in I_j) \leq \frac{\alpha}{B}\]
\end{definition}

\begin{restatable}{lemma}{wellBalanced}
\label{lem:well-balanced}
If $n \geq B\log{\frac{B}{\delta}}$, with probability at least $1 - \delta$, the binning scheme $\mathcal{B}$ we chose is 2-well-balanced.
\end{restatable}

While the way we choose bins is not novel~\cite{zadrozny2001calibrated}, we believe the guarantees around it are---not all binning schemes in the literature allow us to efficiently estimate the calibration error, for example the binning scheme in~\cite{guo2017calibration} does not. Our proof of Lemma~\ref{lem:well-balanced} is in Appendix~\ref{sec:calibrating_models_appendix}. The core challenge is that applying Chernoff bounds or a standard VC dimension argument gives us a loose bound and tells us we need $O(B^2\log{\frac{B}{\delta}})$ samples. We use a discretization argument to prove the result.

In Proposition~\ref{prop:mse-finite-binning} in Appendix~\ref{sec:calibrating_models_appendix}, we also show that if we use many bins, binning the outputs has little impact on model quality as measured by the mean-squared error.

\subsection{Experiments}

Our experiments on CIFAR-10 and ImageNet show that in the low-data regime, for example when we use $\leq 1000$ data points to recalibrate, our variance-reduced calibration method produces models with much lower calibration error than histogram binning. The model's objective was to output a confidence score associated with each class, where we calibrated each class separately as in~\cite{zadrozny2002transforming}, used $B$ bins per class and evaluated calibration using the marginal calibration error (Definition~\ref{dfn:marginal-ce}).

We describe our experimental protocol for CIFAR-10.
The CIFAR-10 validation set has 10,000 data points. We sampled, with replacement, a recalibration set of 1,000 points. We ran either the variance-reduced calibrator (we fit a sigmoid in the function fitting step) or histogram binning and measured the marginal calibration error on the entire set of 10K points.
% \footnote{This is equivalent to using the empirical distribution on the 10K validation points as the true data distribution, and comparing how these methods perform. We do this since we cannot measure the ground truth calibration error.}
We repeated this entire procedure 100 times and computed mean and 90\% confidence intervals, and we repeated this varying the number of bins $B$. Figure~\ref{fig:marginal_calibrator_comparison_cifar} shows that the variance-reduced calibrator produces models with lower calibration error, for example $35\%$ lower calibration error when we use 100 bins per class.

Using more bins allows a model to produce more fine-grained predictions, e.g.~\cite{brocker2012empirical} use $B = 51$ bins, which improves the quality of predictions which can be measured by the mean-squared error -- Figure~\ref{fig:cifar_calibrator_cmp_mse_ce} shows that our method achieves better mean-squared errors for any given calibration constraint.\footnote{By the decomposition of the mean-squared error, this means our method achieves better `sharpness' as well.} More concretely, the figure shows a scatter plot of the mean-squared error and $\lsquared{}$ calibration error for histogram binning and variance-reduced calibration when we vary the number of bins. For example, if we want our models to have an $\lsquared{}$ calibration error $\leq 0.0004$ (which translates to an $\ell_2$ `average' calibration error of 2\%) we get a $9\%$ better mean-squared error. In Appendix~\ref{sec:calibrating_models_appendix_experiments} we show that we can get nearly \emph{5x lower top-label calibration error on ImageNet}, and give further experiment details.
% We also run an ablation to show that if we have more data points then as the theory predicts histogram binning catches up to the variance-reduced calibrator, and that the way we construct bins is crucial, the result do not hold if we use the binning scheme in~\cite{guo2017calibration}.

\begin{figure}
  \centering
  \centering
     \begin{subfigure}[b]{0.54\textwidth}
         \centering
         \includegraphics[width=0.8\textwidth]{marginal_ces.png}
         \caption{Effect of number of bins on $\lsquared{}$ calibration error.}
         \label{fig:marginal_calibrator_comparison_cifar}
     \end{subfigure}
     \hfill
     \begin{subfigure}[b]{0.44\textwidth}
         \centering
         \includegraphics[width=\textwidth]{marginal_mse_vs_ces.png}
         \caption{Tradeoff between calibration and MSE.}
         \label{fig:cifar_calibrator_cmp_mse_ce}
     \end{subfigure}
  \caption{
  (\textbf{Left}) Recalibrating using 1,000 data points on CIFAR-10, our variance-reduced calibrator achieves lower $\lsquared{}$ calibration error than histogram binning, especially when the number of bins $B$ is large.
  (\textbf{Right}) For a fixed calibration error, our variance-reduced calibrator allows us to use more bins. This results in models with more predictive power which can be measured by the mean-squared error. Note the Y-axis range is $[0.4, 0.8]$ to zoom into the relevant region.
  }
  \label{fig:nan2}
\end{figure}


\section{Verifying calibration}
\label{sec:verifying_calibration}

If a model outputs values from some finite set $S$,\footnote{The model can choose the set $S$.} and the probability of outputting each value $s \in S$ is not too small, then we can estimate its calibration error. The plugin estimator requires samples proportional to the number of model outputs $|S|$. Instead, we introduce the \emph{cancelling} estimator that requires samples proportional to $\sqrt{|S|}$. We prove these finite sample guarantees (Theorem~\ref{thm:final-ours}), and show experimental evidence that our estimators approximate the calibration error better. We show, experimentally, that using a better estimator allows us to pick out better models.

The plugin estimator directly estimates each term in the $\ell_2^2$ calibration error from samples. Suppose we wish to measure the calibration error of a model $f : \mathcal{X} \to S$ where $S \subseteq [0, 1]$. Order the elements in $S$: $s_1 < s_2 < \dots < s_B$, where $B = |S|$ denotes the number of model outputs. Suppose we get an evaluation set $T_n = \{(x_1, y_1), \dots, (x_n, y_n)\}$ sampled i.i.d. from $P(X, Y)$.

\begin{definition}[Plugin estimator]
Let $R_i = \{ y_j \; | \; (x_j, y_j) \in T_n\wedge f(x_j) = s_i \}$ denote the label values where the model outputs $s_i$.

We estimate $E[Y | f(X) = s_i]$ as:
\[ \hat{y_i} = \sum_{y \in R_i} \frac{y}{|R_i|} \] 
We estimate $P(f(X) = s_i)$ by:
\[ \hat{p_i} = \frac{|R_i|}{n} \]
Then the plugin estimator for the $\ell_2^2$ error is:
\[ \hat{E}_{\mbox{pl}}^2 = \sum_{i=1}^b \hat{p_i} (s_i - \hat{y_i})^2 \]
\end{definition}

We now define our improved estimator, which debiases the plugin estimator.

\begin{definition}[Cancelling estimator]
The cancelling estimator for the $\ell_2^2$ error is:
\[ \hat{E}^2 = \sum_{i=1}^b \hat{p_i} \Big[ (s_i - \hat{y_i})^2 - \frac{\hat{y_i}(1 - \hat{y_i})}{\hat{p_i}n-1} \Big] \]
\end{definition}

An important condition for verifying calibration is $p_i$ cannot be too small.

\begin{definition}
\label{dfn:low-bal}
We say a model is $k$-lower-balanced if $p(X = s_i) \geq \frac{1}{kB}$ for all $i$.
\end{definition}
 
We now analyze the two estimators. The main results are that to check if the $\ell_2^2$ calibration error of a model is $\leq \epsilon^2$, the plugin estimator requires $\Theta(\frac{b}{\epsilon^2})$ samples while our estimator requires $\theta(\frac{\sqrt{b}}{\epsilon^2})$ samples. See Theorem~\ref{thm:final-plugin} and Theorem~\ref{thm:final-ours} for these main results.

\textbf{We use the following notation simplification} to simplify the theorem statements and proofs:
\[ p_i = P(f(X) = s_i) \]
\[ y_i^* = \mathbb{E}[Y \; | \; f(X) = s_i] \]
\[ e_i = (s_i - y_i^*) \]

Then, if we let ${E^*}^2$ denote the actual $\ell_2^2$ calibration error, we have:
\[ {E^*}^2 = \sum_{i=1}^b p_i e_i^2 \]
\subsection{Analysis of plugin estimator}
\begin{lemma}
\label{lemma:c_n_lemma}
For some $k > 1$, suppose $p_i \geq \frac{1}{kb}$ for all $i$. Then if $n \geq 3kb \log{\frac{2b}{\delta}}$, we have $P(\forall i . \; |p_i - \hat{p_i}| \leq c(n) p_i) \geq 1 - \delta$ where
\[ c(n) = \sqrt{\frac{3kb \log{\frac{2b}{\delta}}}{n}} \]
\end{lemma}

% \begin{lemma}
% The plugin estimator satisfies the following decomposition:
% \[ \hat{E}^2 = \underbrace{\sum_{i=1}^b \hat{p_i}e_i^2}_{(1)}  - \underbrace{2\sum_{i=1}^b \hat{p_i}e_i(\hat{y_i} - y_i^*)}_{(2)} + \underbrace{\sum_{i=1}^b \hat{p_i}(\hat{y_i} - y_i^*)^2}_{(3)} \]
% \end{lemma}

\begin{theorem}
\label{thm:plugin-bound}
For some $k > 1$, suppose $p_i \geq \frac{1}{kb}$ for all $i$. Then for the plugin estimator, if $n \geq 3kb \log{\frac{2b}{\delta}}$, with probability at least $1 - 3\delta$:
\[ | \hat{E_{\mbox{pl}}}^2 - {E^*}^2 | \leq c(n){E^*}^2 + \sqrt{\frac{2(1+c(n)){E^*}^2}{n} \log{\frac{2}{\delta}}} + \frac{b}{2n} \log{\frac{2b}{\delta}} \]
Where $c(n)$ is defined in Lemma~\ref{lemma:c_n_lemma}
\end{theorem}

% Theorem~\ref{thm:plugin-bound} gives a sample complexity bound for the estimation error. In many cases we are interested in checking if our model has calibration error $\leq \epsilon$. In other words, we are given $\epsilon$. If the calibration error is $> \epsilon$, then with probability at least $1 - \delta$ we should output that it is not calibrated. $1-\delta$ is the significance of the test. If the calibration error is $< r\epsilon$, then with probability at least $1 - \delta'$, we should output that the model is calibrated. $1 - \delta'$ is the power at effect size $r < 1$. Typically, we will choose $\delta = \delta'$.
Typically, we are interested in checking if our model has calibration error $\leq \epsilon$. In other words, we are given $\epsilon, \delta > 0$ and effect size $r < 1$. If the calibration error is $> \epsilon$, then with probability at least $1 - \delta$ we should output that it is not calibrated. If the calibration error is $< r\epsilon$, then with probability at least $1 - \delta$, we should output that the model is calibrated. Another way of thinking about this is that we want to estimate the calibration error within a constant multiplicative factor.

\begin{theorem}
\label{thm:final-plugin}
Using the plugin estimator $\hat{E}_{\mbox{pl}}$, if $n = \Theta(kb + \frac{b}{\epsilon^2})$ ignoring $\log$ factors, we can check if ${E^*}^2 \leq \epsilon^2$ with failure probability $\delta$, and constant effect size $r$. 
\end{theorem}

% \begin{corollary}
% \label{cor:final-plugin}
% Using the plugin estimator $\hat{E}_{\mbox{pl}}$, if $n = \Theta(kb + \frac{b}{\epsilon^2})$ ignoring $\log$ factors, we can check if $|{E^*} | \leq \epsilon$ with significance and power $\delta$, and constant effect size $r$. 
% \end{corollary}

\subsection{Analysis of our estimator}

\begin{theorem}
\label{thm:our-bound}
For some $k > 1$, suppose $p_i \geq \frac{1}{kb}$ for all $i$. Then for the plugin estimator, if $n \geq 3kb \log{\frac{2b}{\delta}}$, with probability at least $1 - 4\delta$:
\[ | \hat{E}^2 - {E^*}^2 | \leq c(n){E^*}^2 + \sqrt{\frac{2(1+c(n)){E^*}^2}{n} \log{\frac{2}{\delta}}} + \frac{1}{n} + \frac{\sqrt{b}}{n}\log{\frac{4n}{\delta}} \log{\frac{2}{\delta}}\]
Where $c(n)$ is defined in Lemma~\ref{lemma:c_n_lemma}
\end{theorem}

\begin{theorem}
\label{thm:final-ours}
Using our estimator $\hat{E}$, if $n = \Theta(kb + \frac{\sqrt{b}}{\epsilon^2})$ ignoring $\log$ factors, we can check if ${E^*}^2  \leq \epsilon^2$ with failure probability $\delta$, and constant effect size $r$. 
\end{theorem}

% This means that our estimator has a substantially better dependency on the number of outputs of the model.

% \begin{corollary}
% \label{cor:final-ours}
% Using our estimator $\hat{E}$, if $n = \Theta(kb + \frac{\sqrt{b}}{\epsilon^2})$ ignoring $\log$ factors, we can check if $|{E^*} | \leq \epsilon$ with significance and power $\delta$, and constant effect size $r$. 
% \end{corollary}

\section{Additional related work}

Calibration, including the $\ell_2^2$ calibration error, has been studied in many fields such as meteorology~\cite{murphy1973vector, murphy1977reliability, degroot1983forecasters,gneiting2005weather, brocker2009decomposition}, medicine~\cite{jiang2012calibrating, crowson2017calibration, harrell1996prognostic}, reinforcement learning~\cite{malik2019calibrated}, natural language processing~\cite{nguyen2015posterior, card2018calibration}, speech recognition~\cite{dong2011calibration}, econometrics~\cite{gneiting2007probabilistic}, psychology~\cite{lichtenstein1982calibration}, and machine learning~\cite{guo2017calibration, zadrozny2001calibrated, kuleshov2018accurate, zadrozny2002transforming, naeini2014binary, hendrycks2019anomaly, hendrycks2019pretraining}. Besides the calibration error metric, prior work also uses the Hosmer-Lemeshov test~\cite{hosmer1980goodness} and reliability diagrams~\cite{degroot1983forecasters, brocker2007reliability} to evaluate calibration. Besides calibration, other ways of producing and quantifying uncertainties include Bayesian methods~\cite{gelman1995bayesian} and conformal prediction~\cite{shafer2008tutorial, lei2016distribution}.

Recalibration is related to (conditional) density estimation~\cite{wasserman2019, parzen1962} as the goal is to estimate $\expect[Y \mid f(X)]$. The $L_2$ loss function used in the density estimation literature is precisely the $\ell_2^2$ calibration error. Algorithms and analysis in density estimation typically assume the true density is $L-$Lipschitz, while in calibration applications, the calibration error of the final model should be measurable from data, without making untestable assumptions on $L$---this leads to different techniques and analysis.
% whereas in calibration applications we want to bound our calibration error without making unverifiable assumptions on $L$


Bias is a common issue with statistical estimators, for example for the sample standard deviation. It has also long been known that the mean-squared error, measured on samples, gives a biased estimate---the seminal work by Stein~\cite{stein81sure} investigates and fixes this bias. However, debiasing an estimator does not typically lead to \emph{an improved sample complexity}, as it does in our case.

\pl{first impression is that this looks really brief...you can say calibration has been studied traditionally in many fields: meteorology (cite), ML (cite), etc.;
then is there's probably non calibration work that's related...just non-parametric function estimation...how is calibration different?
anything related for proof techniques?
}

\section{Conclusion}

In this paper we had three contributions: 1. We showed that the calibration error of popular continuous methods is underestimated; 2. We introduced the first method, to our knowledge, that has better sample complexity than histogram binning but has a \emph{measurable calibration error}, giving us the best of both worlds of scaling and binning; and 3. We showed that an alternative estimator has better sample complexity than the commonly used plugin estimator. Our method gives us 35\% lower calibration error on CIFAR-10, and up to 5x lower calibration error on ImageNet, with $B = 100$ bins.

\section{Acknowledgements}

The authors would like to thank the Open Philantropy Project, Stanford Graduate Fellowship, and Toyota Research Institute for funding. Toyota Research Institute (``TRI") provided funds to assist the authors with their research but this article solely reflects the opinions and conclusions of its authors and not TRI or any other Toyota entity.

We are grateful to Pang Wei Koh, Chuan Guo, Anand Avati, Shengjia Zhao, Weihua Hu, Yu Bai, John Duchi, Dan Hendrycks, Jonathan Uesato, Michael Xie, Albert Gu, Aditi Raghunathan, Fereshte Khani, Stefano Ermon, Eric Nalisnick, and Pushmeet Kohli for insightful discussions. We thank the anonymous reviewers for their thorough reviews and suggestions that have improved our paper. We would also like to thank Pang Wei Koh, Yair Carmon, Albert Gu, Rachel Holladay, and Michael Xie for their inputs on our draft, and Chuan Guo for providing code snippets from their temperature scaling paper.

\newpage
% We hope future work looks at even stronger notions of multiclass calibration (prior work typically focuses on top-label calibration) and calibration under domain shift.

% We think our framework opens up many new avenues for exploration. Can we come up with a binning scheme that is better than the well-balanced binning scheme that is one that leads estimate the calibration error even faster, at least for 

% \tm{I guess we need a conclusion section. }

% \tm{briefly summarize the contribution (now we can use slightly different language because we assume the readers have read most of the paper and now the definitions.)}
% \tm{Open question or future directions, potential implication of the paper}



\bibliographystyle{unsrtnat}
\bibliography{local,refdb/all}

\appendix

\newpage 
\section{Model and code details}

The VGG16 model we used for ImageNet experiments is from the Keras~\cite{chollet2015} module in the TensorFlow~\cite{tensorflow2015-whitepaper} library and we used pre-trained weights supplied by the library. The VGG16 model for CIFAR-10 was obtained from an open-source implementation on GitHub~\cite{geifman2017}, and we used the pre-trained weights there. We independently verified the accuracies of these models.

We include our code for the experiments in the supplementary material.

\newpage
\section{Ablations for Section~\ref{sec:challenges-measuring}}
\label{sec:appendix_platt_experiments}

Here we present additional experiments for Section~\ref{sec:challenges-measuring}.
Recall that the experiments in section~\ref{sec:challenges-measuring} showed that binning underestimates the calibration error of a model---we focused on the $\ell_2\mbox{-CE}$ and selected bins so that each bin has an equal number of data points. Figure~\ref{fig:imagenet_lower_bound_l1} shows that binning is also unreliable at measuring the $\ell_1\mbox{-CE}$ (ECE) on ImageNet---using more bins uncovers a higher calibration error than we would otherwise detect with fewer bins. Figure~\ref{fig:imagenet_lower_bound_l1_prob} shows that the same conclusion holds on ImageNet if we look at the $\ell_1\mbox{-CE}$ \emph{and} use an alternative approach to selecting bins used in~\cite{guo2017calibration} that we call \emph{equal-width binning}. Here, the $B$ bins are selected to be $I_1 = [0, \frac{1}{B}], I_2 = (\frac{1}{B}, \frac{2}{B}], \dots, I_B = (\frac{B-1}{B}, 1]$. The experimental protocol is the same as in section~\ref{sec:challenges-measuring}.

\begin{figure}
     \centering
     \begin{subfigure}[b]{0.45\textwidth}
         \centering
         \includegraphics[width=\textwidth]{l1_lower_bound_imagenet_plot}
         \caption{ImageNet, $\ell_1\mbox{-CE}$}
         \label{fig:imagenet_lower_bound_l1}
     \end{subfigure}
     \hfill
     \begin{subfigure}[b]{0.45\textwidth}
         \centering
         \includegraphics[width=\textwidth]{l1_lower_bound_imagenet_plot_prob_bin}
         \caption{ImageNet, $\ell_1\mbox{-CE}$, equal-width binning}
         \label{fig:imagenet_lower_bound_l1_prob}
     \end{subfigure}
        \caption{
        Binned $\ell_1$ calibration errors of a recalibrated VGG-net model on ImageNet with $90\%$ confidence intervals. The binned calibration error increases as we increase the number of bins. This suggests that binning cannot be reliably used to measure the $\ell_1\mbox{-CE}$.
        }
        \label{fig:lower_bounds_l1_imagenet}
\end{figure}

We repeated both of these experiments on CIFAR-10 as well, and plot the results in Figure~\ref{fig:lower_bounds_l1_cifar}. Here the results are inconclusive because the error bars are large. This is because the CIFAR-10 dataset is smaller than ImageNet, and the accuracy of the CIFAR-10 model is 93.1\%, so the calibration error that we are trying to measure is much smaller.

\begin{figure}
     \centering
     \begin{subfigure}[b]{0.45\textwidth}
         \centering
         \includegraphics[width=\textwidth]{l1_lower_bound_cifar_plot}
         \caption{CIFAR-10, $\ell_1\mbox{-CE}$}
         \label{fig:cifar_lower_bound_l1}
     \end{subfigure}
     \hfill
     \begin{subfigure}[b]{0.45\textwidth}
         \centering
         \includegraphics[width=\textwidth]{l1_lower_bound_cifar_plot_prob_bin}
         \caption{CIFAR-10, $\ell_1\mbox{-CE}$, equal-width binning}
         \label{fig:cifar_lower_bound_l1_prob}
     \end{subfigure}
        \caption{
        Binned $\ell_1$ calibration errors of a recalibrated VGG-net model on CIFAR-10 with $90\%$ confidence intervals. The results are not as conclusive here because the error bars are large, however it seems to suggest that the binned calibration error increases as we increase the number of bins.
        }
        \label{fig:lower_bounds_l1_cifar}
\end{figure}

We provide details on the dataset split for CIFAR-10. For CIFAR-10, we used a VGG16 model and split the test set into 3 sets of size $(1000, 1000, 8000)$, where used the first set of data to recalibrate the model using Platt scaling, the second to select the binning scheme, and the third to measure the binned calibration error. As stated in the main body of the paper, for ImageNet we used a split of $(20000, 5000, 25000)$.


\newpage
\section{Proofs for section~\ref{sec:challenges-measuring}}
\label{sec:appendix-platt-not-calibrated}

% \begin{example}
% For any binning scheme $\mathcal{B}$ and $p \in \mathbb{Z}^+$, there exists a distribution $P$ over $X, Y$ and a function f s.t. $\lpce(f, B) = 0$ but $\lpce(f, B) = 0.5$. Note that the $\lpce$ is always between 0 and 1.
% \end{example}

% The proofs for Section~\ref{sec:challenges-measuring} unlike for the other sections, from a technical standpoint, fairly standard and not a contribution of our paper. We give them only for completeness.

\continuousNotCalibrated*

\begin{proof}
As stated in the main text, the intuition is that in each interval $I_j$ in $\bins{}$, the model could underestimate the true probability $\expect[Y \mid f(X)]$ half the time, and overestimate the probability half the time. So if we average over the entire bin the model appears to be calibrated, even though it is very uncalibrated. The proof simply formalizes this intuition.

Since $f$ is bijective and continuous we can select data distribution $P$ s.t. $f(X) \sim \mbox{Uniform}[0.5 - \epsilon, 0.5 + \epsilon]$ for any $\epsilon > 0$. To see this, first note that from real analysis since $f : [0, 1] \to [0, 1]$ and $f$ is bijective and continuous, $f^{-1}$ is also bijective and continuous.
Then we can let $X \sim f^{-1}(\mbox{Uniform}[0.5 - \epsilon, 0.5 + \epsilon])$ which has the desired property and has density.

Now, consider each interval $I_j$ in binning scheme $\bins{}$, and let $A_j = I_j \cap \mbox{Uniform}[0.5 - \epsilon, 0.5 + \epsilon]$.
If $A_j = \emptyset$ then $P(f(X) \in A_j) = 0$ so we can ignore this interval (since $f(X)$ will never land in this bin).
Let $p_j = \expect[f(X) \mid f(X) \in A_j]$.
Note that $\expect[f(X) \mid f(X) \in A_j] = \expect[f(X) \mid f(X) \in I_j]$.
Since $f(X) \in [0.5 - \epsilon, 0.5 + \epsilon]$, $p_j \in [0.5 - \epsilon, 0.5 + \epsilon]$ as well.
We will choose $P(Y)$ so that $Y$ is $1$ whenever $f(X)$ lands in the first $p_j$ fraction of interval $A_j$, and $0$ whenever $f(X)$ lands in the latter $1 - p_j$ fraction of $A_j$.
Then $\expect[Y \mid f(X) \in A_j] = p_j$, so the binned calibration error is 0.
But notice that for all $s \in [0.5 - \epsilon, 0.5 + \epsilon]$, $\expect[Y \mid f(X) = s]$ is either $0$ or $1$.
So we have:
\[ \lvert \expect[Y \mid f(X) = s] - s \rvert \geq 0.5 - \epsilon \]
That is, at every point the model is actually very miscalibrated at each $s$. By taking $\epsilon$ very small, we then get that $\lpce(p) \geq 0.5 - \epsilon'$ for any $\epsilon' > 0$, which completes the proof.
\end{proof}


\binningLowerBound*

\begin{proof}
It suffices to prove the claim for the $\ell_p^p$ error:
\[ (\lpce(f_{\bins{}}))^p \leq (\lpce(f))^p \]
This is because if $p > 0$ then $a \leq b \Leftrightarrow a^p \leq b^p$.

For $p \geq 1$, let $l(a, b) = (|a - b|)^p$.
We note that $l$ is convex in both arguments.
The proof is now a simple result of Jensen's inequality and convexity of $l$.
Suppose that $\mathcal{B}$ is given by intervals $I_1, ..., I_B$.
Let $Z = f(X)$---note that $Z$ is a random variable.

We can write $(\lpce(f_{\bins{}}))^p$ as:
\[ (\lpce(f_{\bins{}}))^p = \sum_{j=1}^B P(Z \in I_j) \; l\Big( \mathbb{E}[Z \mid Z \in I_j], \mathbb{E}[Y \mid Z \in I_j] \Big) \]
We can write $(\lpce(f))^p$ as:
\[ (\lpce(f))^p = \sum_{j=1}^B P(Z \in I_j) \; \mathbb{E}\Big[ l\big( Z, \mathbb{E}[Y \mid Z] \big) \mid Z \in I_j \Big] \]
Fix some bin $I_j \in \bins{}$. By Jensen's inequality,
\[ l\Big( \mathbb{E}[Z \mid Z \in I_j], \mathbb{E}[Y \mid Z \in I_j] \Big) \leq \mathbb{E}\Big[ l\big( Z, \mathbb{E}[Y \mid Z] \big) \mid Z \in I_j \Big] \]
Since this inequality holds for each term in the sum, it holds for the whole sum:
\[ (\lpce(f_{\bins{}}))^p \leq (\lpce(f))^p \]
Note that the proof also implies that finer binning schemes give a better lower bound.
That is, given $\bins{}'$ suppose for all $I_j' \in \bins{}'$, $I_j' \subseteq I_k$ for some $I_k \in \bins{}$.
Then $\lpce(f_{\bins{}}) \leq \lpce(f_{\bins{}'}) \leq \lpce(f)$.
This is because $f_{\bins{}'}$ can be seen as a binned version of $f_{\bins{}}$.
% Fix bin $I_j \in \bins{}$.
% Since $\bins{}'$ is finer than $\bins{}$, there exists some $S_j$ s.t. $I_j = \bigcup_{s \in S_j} I_s'$.

% Define the following notation to simplify the proof.
% \[ p_s = P(Z \in I_s' | Z \in I_j) \]
% \[ f_s = \mathbb{E}[Z | Z \in I_s']\]
% \[ y_s = \mathbb{E}[Y | Z \in I_s']\]
% Since $I_j = \bigcup_{s \in S_j} I_s'$ and the intervals are disjoint, we have:
% \[ \sum_{s \in S_j} p_s = 1 \]
% We can write $(\lpce(f, \bins{}))^p$ and $(\lpce(f_{\bins{}'}))^p$ as a weighted sum of the errors in each bin $I_j$.
% \[ (\lpce(f, \bins{}))^p = \sum_{j=1}^m P(Z \in I_j) \; l\Big( \big( \sum_{s \in S_j} p_s f_s \big), \big( \sum_{s \in S_j} p_s y_s \big) \Big) \]
% \[ (\lpce(f_{\bins{}'}))^p = \sum_{j=1}^m P(Z \in I_j) \Big( \sum_{s \in S_j} p_s l(f_s, y_s) \Big) \]
% Then by Jensen's inequality, from the convexity of $l$,
% \[ l\Big( \big( \sum_{s \in S_j} p_s f_s \big), \big( \sum_{s \in S_j} p_s y_s \big) \Big) \leq \sum_{s \in S_j} p_s l(f_s, y_s)  \]
% Since this inequality holds for each term in the sum, it holds for the whole sum:
% \[ (\lpce(f, \bins{}))^p \leq (\lpce(f_{\bins{}'}))^p \]

% The proof of the second inequality, that $(\lpce(f_{\bins{}'}))^p \leq (\lpce(f))^p$, closely parallels the first.
\end{proof}


\newpage
\section{Proofs for section~\ref{sec:calibrating_models}}
\label{sec:calibrating_models_appendix}

\newcommand{\G}[0]{\ensuremath{\mathcal{G}}}

Our analysis of the sample complexity of \ourcal{} requires some assumptions on the function family $\G{}$ that hold for standard methods like Platt scaling, beta calibration, and vector scaling (since vector scaling calibrates each class separately):

\begin{enumerate}
\item (Finite bounded parameters). Let $\G{} = \{ g_{\theta} : \mathcal{Z} \to [0, 1] \; | \; \theta \in \mathbb{R}^{d'} \wedge ||\theta||_{\infty} \leq R \}$
\item (Injective). For all $g_{\theta} \in \G{}$ we assume $g_{\theta}$ is injective.
\item (Lipschitz). Suppose that for all $z \in \mbox{supp}(Z)$, $|g_{\theta_1}(z) - g_{\theta_2}(z)| \leq L|\theta_1 - \theta_2|_2$.
\end{enumerate}

If $\G{}$ satisfies the following assumptions we say $\G{}$ is $(d', L, R)$-compatible. For Platt-scaling $d' = 1$ and $L$ and $R$ are typically small constants.

We will analyze each step of our algorithm and then combine the pieces to get Theorem~\ref{thm:final-calib}.
As we mention in the main text, step 3 is the main variance-reduction step, so Lemma~\ref{thm:empirical-binning} is one of the core parts of our proof.
Step 2 is where we construct a binning scheme so that each bin has an equal number of points---we show that this property holds approximately in the population.
This is important as well, particularly to ensure we can estimate the calibration error.
Step 1 is a simple adaptation of a standard result from learning theory.

% We first need to introduce the Bayes optimal recalibrator $\omega^*$, which we compare all other recalibrators to. $\omega^* \circ f$ has calibration error $0$, and also has the minimum mean-squared error among all recalibrators.

% \begin{definition}
% $\omega^* : \mathcal{Z} \to [0, 1]$ is given by $\omega^*(x) = \expect[Y | Z = z]$.
% \end{definition}

We first show a generalization result for Step 1 of our approach---recall that step 1 is essentially Platt scaling.
That is, if $\G{}$ is $(d', L, R)$-compatible and contains $g^* \in \G{}$ with low calibration error, then we show that the empirical risk minimizer $g \in \G{}$ of the mean-squared loss will also quickly converge to low calibration error.
Intuitively, methods like Platt scaling fit a single parameter to the data so classical learning theory bounds tell us they will converge quickly to their optimal error, at least in mean-squared error.
We can combine this with a decomposition of the mean-squared error into calibration and refinement, and the injectivity of $g \in \mathcal{G}$, to show they also converge in calibration error.

\begin{lemma}
There exists a constant $c$, independent of $d', L, n$ such that with high probability over the recalibration samples $\ltwoce(g) \leq \min_{g' \in \G{}}\ell_2^2\mbox{-CE}(g') + \frac{cLd' \log{R}}{\sqrt{n}}$, where recall that $g \in \G{}$ was selected as the empirical risk minimizer of the mean-squared error loss.
\end{lemma}

\begin{proof}
We use the classic decomposition of the mean-squared error into calibration error (also known as reliability) and refinement\footnote{Note that the refinement term can be further decomposed into resolution (also known as sharpness) and irreducible uncertainty.}. For any $g \in \G{}$ we have:
\[ \mbox{MSE}(g) = \underbrace{\ltwoce(g)}_{\mbox{calibration}} + \underbrace{\expect[(\expect[Y \mid g(Z)] - Y)^2]}_{\mbox{refinement}} \]
Note that the refinement term is constant for all injective $g \in \G{}$, since for injective $g$:
\[ \expect[(\expect[Y \mid g(Z)] - Y)^2] = \expect[(\expect[Y \mid Z] - Y)^2] \]
This means that the difference in $\mbox{MSE}$ between any $g$ and $g'$ is precisely the difference in the mean-squared error. From a standard $\epsilon$-cover in the parameter space, or PAC-Bayes argument, we can get a high probability generalization bound on the mean-squared error:
\[ \mbox{MSE}(g) \leq \mbox{MSE}(g^*) + \frac{cLd' \log{R}}{\sqrt{n}} \]
This gives us the desired result.
\end{proof}

Our next lemma will require showing convergence in $\ell_2$ and $\ell_1$ norm in function space, which we define below:

\begin{definition}
Given $f, g : \mathcal{Z} \to [0, 1]$, for the $\ell_2$ norm we define $||f - g||_2^2 = \expect[(f(Z) - g(Z))^2]$ and $||f- g||_2 = \sqrt{||f - g||_2^2}$. For the $\ell_1$ norm we define $||f - g||_1 = \expect[\lvert f(Z) - g(Z)\rvert]$
\end{definition}

Recall that we showed that in the limit of infinite data the binned version of $g$, $g_{\bins{}}$, has lower calibration error than $g$ (Proposition~\ref{prop:bin_low_bound}). However our method uses $n$ data points to empirically bin $g$, giving us $\hat{g_{\bins{}}}$. We now show the key lemma that allows us to bound the calibration error and later the mean-squared error. That is, we show that the empirically binned function $\hat{g_{\bins{}}}$ quickly converges to $g_{\bins{}}$ in both $\ell_2$ and $\ell_1$ norms.

\begin{lemma}
\label{thm:empirical-binning}
There exist constants $c_B, c_1, c_2$ such that the following is true. Given $g : \mathcal{Z} \to [0, 1]$, binning set $T_3 = \{(z_i, y_i)\}_{i=1}^n$ and a 2-well-balanced binning scheme $\bins{}$ of size $B$. Given $0 < \delta < 0.5$, suppose that $n \geq c_B B\log{\frac{B}{\delta}}$. Then with probability at least $1 - \delta$,  $||\hat{g_{\bins{}}} - g_{\bins{}}||_2 \leq \frac{c_2}{\sqrt{n}}\sqrt{\log{\frac{B}{\delta}}}$ and $||\hat{g_{\bins{}}} - g_{\bins{}}||_1 \leq \frac{c_1}{\sqrt{nB}}\sqrt{\log{\frac{B}{\delta}}}$
\end{lemma}

\begin{proof}
% Note, for the L2 bound we only need 1 side of 2-well-balanced, P(f(X) \in I_j) \geq 1/2B for all j.
% For the L1 bound, if we have only 1 side of 2-well-balanced, but b_j - b_{j-1} are equal for all j, then we get the same bound.
% One question is whether we can relax this theorem, for example if b_j - b_{j-1} are equal for all j. Maybe we don't need well-balancedness in that case.
% We first discuss high level intuition for the proof. Recall that in the naive binning approach, where we bin the label (that is, $Y$) values, the $\ell_2$ calibration error is $O(\sqrt{\frac{B}{n}})$ ignoring $\log$ factors. That is, the number of samples we need for calibration increases linearly with the number of bins. The intuition is that when we have more bins, we have fewer samples per bin. That is, we have $\frac{n}{B}$ samples of $Y$ in each bin, where $Y \in \{0, 1\}$. Then Hoeffding's bound gives us the above result, which is tight up to constants. However, this theorem shows when we bin $f$ instead of $Y$, our rates have much better dependencies on $B$. The high level idea is that for each bin $j$ we are taking the average of values bounded in $I_j = [b_{j-1}, b_j]$. This is a narrower range than in naive binning where we took the average of values in $\{0, 1\}$. Since the values are bounded in a narrower range, the variance is lower. Of course, the variance of any particular bin could be big. For example we could have $I_7 = [0.1, 0.9]$. But the sum of all the bin sizes is 1, so `most' of the bins have a lower variance. We now give the formal proof.

Recall that the intuition is in Figure~\ref{fig:variance_reduced_illustration} of the main text---the $g(z_i)$ values in each bin (gray circles in Figure~\ref{fig:var_red_binning}) are in a narrower range than the $y_i$ values (black crosses in Figure~\ref{fig:hist_binning}) so when we take the average we incur less of an estimation error. Now, there may be a small number of bins where the $g(z_i)$ values are not in a narrow range, but we will use the assumption that $\bins{}$ is 2-well-balanced to show that these effects average out and the overall estimation error is small.

Define $R_j$ to be the set of $g(z_i)$ that fall into the $j$-th bin, given by $R_j = \{g(z_i) \mid g(z_i) \in I_j \wedge (z_i, y_i) \in T_3\}$ (recall that $T_3$ is the data we use in step 3).
Let $p_j$ be the probability of landing in bin $j$, given by $p_j = \prob(g(Z) \in I_j)$.
Since $\bins{}$ is 2-well-balanced, $p_j \geq \frac{1}{2B}$.
Since $n \geq c_B B\log{\frac{B}{\delta}}$, by the multiplicative Chernoff bound, for some large enough $c_B$, with probability at least $1 - \frac{\delta}{2}$, $|R_j| \geq \frac{p_j}{2}$.

Consider each bin $j$. Let $\mu_j$ be the expected output of $g$ in bin $j$, given by $\mu_j = \expect[g(Z) \; | \; g(Z) \in I_j]$. $\mu(R_j)$, the mean of the values in $R_j$, is the empirical average of $|R_j|$ such values, each bounded between $b_{j-1}$ and $b_j$ where $I_j = [b_{j-1}, b_j]$. So $\hat{\mu}(R_j)$ is sub-Gaussian with parameter:

\[ \sigma^2 = \frac{(b_j - b_{j-1})^2}{4|R_j|} \leq \frac{(b_j - b_{j-1})^2}{2p_jn} \]

Then by the sub-Gaussian tail bound, for any $1 \leq j \leq B$, with probability at least $1 - \frac{\delta}{2B}$, we have:
\[ (\mu_j - \hat{\mu}(R_j))^2 \leq \frac{(b_j - b_{j-1})^2}{p_jn} \log{\frac{4B}{\delta}} \] 

So by union bound with probability at least $1 - \frac{\delta}{2}$ the above holds for all $1 \leq j \leq B$ simultaneously.

We then bound the $\ell_2$-error.
\begin{align*}
||\hat{f_{\mathcal{B}}} - f_{\mathcal{B}}||_2 &= \sqrt{\sum_{j =1}^B p_j (\mu_j - \hat{\mu}(R_j))^2} \\
&\leq \sqrt{\sum_{j =1}^B p_j \frac{(b_j - b_{j-1})^2}{p_jn} \log{\frac{4B}{\delta}}} \\
&\leq \sqrt{\frac{1}{n} \log{\frac{4B}{\delta}} \sum_{j =1}^B (b_j - b_{j-1})^2 } \\
&\leq \sqrt{\frac{1}{n} \log{\frac{4B}{\delta}} \sum_{j =1}^B (b_j - b_{j-1}) } \\
&\leq \sqrt{\frac{1}{n} \log{\frac{4B}{\delta}} } \\
&\leq c_2 \frac{1}{\sqrt{n}} \sqrt{\log{\frac{B}{\delta}}}
\end{align*}

Similarly, we can also bound the $\ell_1$-error. Here we also use the fact that $p_j \leq \frac{2}{B}$ since $\bins{}$ is 2-well-balanced.
\begin{align*}
||\hat{f_{\mathcal{B}}} - f_{\mathcal{B}}||_1 &= \sum_{j =1}^B p_j |\mu_j - \hat{\mu}(R_j)| \\
&\leq \sum_{j =1}^B p_j \sqrt{\frac{(b_j - b_{j-1})^2}{p_jn} \log{\frac{4B}{\delta}}} \\
&\leq \sum_{j =1}^B \sqrt{\frac{p_j(b_j - b_{j-1})^2}{n} \log{\frac{4B}{\delta}}} \\
&\leq \sum_{j =1}^B \sqrt{\frac{2(b_j - b_{j-1})^2}{Bn} \log{\frac{4B}{\delta}}} \\
&\leq \sqrt{\frac{2}{Bn} \log{\frac{4B}{\delta}}} \sum_{j =1}^B (b_j - b_{j-1}) \\
&\leq c_1 \frac{1}{\sqrt{Bn}} \sqrt{\log{\frac{B}{\delta}}}
\end{align*}
By union bound, these hold with probability at least $1 - \delta$, which completes the proof.
\end{proof}

In our proofs we also required that the binning scheme we constructed was 2-well-balanced.

\begin{lemma}
There exists constant $c$ s.t. if $n \geq c B \log{\frac{B}{\delta}}$ then the binning scheme $\bins{}$ we construct in step 2 of our algorithm is 2-well-balanced.
\end{lemma}

\begin{proof}
This can be proved using a PAC-Bayes relative error bound.
\end{proof}

Finally, we have the tools to prove the main theorem, which we restate below.

\finalCalib*{}

\begin{proof}
The proof simply pieces together Lemma X, Y, Z.
If $n = \Theta(\frac{(L'd'\log{R})^2}{\epsilon^2})$, from Lemma X, step 1 gives us $g$ with $\ltwoce(g) \leq \epsilon^2$.
Next if $n = \Theta(B \log{\frac{B}{\delta}})$, from Lemma Y, step 2 chooses a 2-well-balanced binning scheme.
From Proposition Z, $\ltwoce(g_{\bins{}}) \leq \ltwoce(g)$.
From Lemma W, if $n = \frac{1}{\epsilon^2} \log{\frac{B}{\delta}}$, step 3 gives us $\hat{g_{\bins{}}}$ with $||\hat{g_{\bins{}}} - g_{\bins{}}||_2 \leq \epsilon$.

% From proposition Y, $\twoce(g_{\bins{}}) \leq \ltwoce(g)$
\end{proof}

We also show that if we use lots of bins, discretization has little impact on model quality as measured by the mean-squared error.

\begin{theorem}[MSE bound]
\label{thm:sharpness-bound}
If $\mathcal{B}$ is a 2-well-balanced binning scheme of size $B$ and $B \leq O(n\log{n})$, then with high probability $\mbox{MSE}(\hat{g}_{\mathcal{B}}) \leq \mbox{MSE}(g) + O(\frac{1}{B})$.
\end{theorem}






















\section{Experimental details and ablations for section~\ref{sec:calibrating_models}}

\label{sec:calibrating_models_appendix_experiments}

We give more experimental details for our CIFAR-10 experiment, show experimental results for top-label calibration in ImageNet and CIFAR-10, and give details and results for our synthetic experiments. Note that the code is available in the supplementary folder for completeness.

\textbf{Experimental details}: We detail our experimental protocol for CIFAR-10 first. The CIFAR-10 validation set has 10,000 data points. We sampled, with replacement, a recalibration set of 1,000 points. In our theoretical approach and analysis, we split up these sets into multiple parts. For example, we used the first part for training a function, second part for bin construction, third part for binning. In practice, using the same set for all three steps worked out better, for both histogram binning and \ourcal{}. We believe that there may be theoretical justification for merging these sets, although we leave that for future work. For the marginal calibration experiment we ran either \ourcal{} (we fit a sigmoid in the function fitting step) or histogram binning. We calibrated each of the $K$ classes seprately as described in Section~\ref{sec:formulation}, and measured the marginal calibration error on the entire set of 10K points. We repeated this entire procedure 100 times, and computed mean and 90\% confidence intervals.

\begin{figure}
  \centering
  \centering
  	 \begin{subfigure}[b]{0.48\textwidth}
         \centering
         \includegraphics[width=\textwidth]{top_ces_imagenet.png}
         \caption{Effect of number of bins $B$ on top calibration error $\topsquaredce$ on ImageNet.
         }
         \label{fig:imagenet_top_cal_var_red}
     \end{subfigure}
     \hfill
     \begin{subfigure}[b]{0.48\textwidth}
         \centering
         \includegraphics[width=\textwidth]{top_ces_cifar_1000.png}
         \caption{Effect of number of bins $B$ on top calibration error $\topsquaredce$ on CIFAR-10.
         }
         \label{fig:cifar_top_cal_var_red}
     \end{subfigure}
  \caption{
    Recalibrating using 1,000 data points on ImageNet and CIFAR-10, \ourcal{} typically achieves lower $\lsquared$ calibration error than histogram binning, especially when the number of bins $B$ is large. The difference is very significant on ImageNet, where our method does better when $B \geq 10$, and gets a nearly 5 times lower calibration error when $B = 100$. For CIFAR-10 our method does better when $B > 30$, which supports the theory, which predicts that our method does better when $B$ is large. However, when $B$ is small, practitioners should try both histogram binning and \ourcal{}.
    \pl{still need to update the legend to scaling-binning}
}
  \label{fig:mse_estimators_bins}
\end{figure}

In this experiment, we are checking a very precise hypothesis---assuming that the empirical distribution on the 10,000 validation points is the true data distribution, how do these methods perform? This is similar to the experimental protocol used in e.g.~\cite{brocker2012empirical}.
% This is not the same as an alternative hypothesis---how do these methods perform on the true CIFAR-10 distribution, which we do not have access to.
An alternative experimental protocol would have been to first split the CIFAR-10 data into two sets of size $(1000, 9000)$.
We could have then used the first set to recalibrate the model using either \ourcal{} or histogram binning, and then used the remaining 9,000 examples to estimate the calibration error on the ground truth distribution, using Bootstrap to compute confidence intervals.
However, when we ran this experiment, we noticed that the results were very sensitive to which set of 1,000 points we used to recalibrate.
Multiple runs of this experiment led to very different results.
The point is that there are two sources of randomness---the randomness in the data the recalibration method operates on, and the randomness in the data used to evaluate and compare the recalibrators.
In our protocol we account for both of these sources of randomness.

\textbf{Top-label calibration experiments}: We also ran experiments on top-label calibration, for both ImageNet and CIFAR-10. The protocol is exactly as described above, except instead of calibrating each of the $K$ classes, we calibrated the top probability prediction of the model. More concretely, for each input $x_i$, the uncalibrated model outputs a probability $p_i$ corresponding to its top prediction $k_i$, where the true label is $y_i$. We create a new dataset $\{(p_1, \mathbb{I}(k_1 = y_1)), \dots, (p_n, \mathbb{I}(k_n = y_n))\}$ and run \ourcal{} (fitting a sigmoid in the function fitting step, as in Platt scaling) or histogram binning on this dataset, using $B$ bins. This calibrates the probability corresponding to the top prediction of the model. We evaluate the recalibrated models on the top-label calibration error metric ($\topsquaredce{}$) described in Section~\ref{sec:formulation}. For both CIFAR-10 and ImageNet we sampled, with replacement, a recalibration set of 1,000 points for the recalibration data, and we measured the calibration error on the entire set (10,000 points for CIFAR-10, and 50,000 points for ImageNet) as above. We show $90\%$ confidence intervals for all plots.

Figure~\ref{fig:imagenet_top_cal_var_red} shows that on ImageNet \ourcal{} gets significantly lower calibration errors than histogram binning when $B \geq 10$, and nearly a 5 times lower calibration error when $B = 100$. Both methods get similar calibration errors when $B = 1$ or $B = 5$. Figure~\ref{fig:cifar_top_cal_var_red} shows that on CIFAR-10 when $B$ is high, \ourcal{} gets lower calibration errors than histogram binning, but when $B$ is low histogram binning gets lower calibration errors. We believe that the difference might be because the CIFAR-10 model is highly accurate at top-label prediction to begin with, getting an accuracy of over $93\%$, so there is not much scope for re-calibration. In any case, this ablation tells us that practitioners should try multiple methods when recalibrating their models and evaluate their calibration error.

\textbf{(A) Synthetic experiments to validate bounds}: Suppose the model, before recalibration, outputs values in $[0, 1]$, that is, $\mathcal{Z} = [0, 1]$. We first describe the scaling family we use, which is Platt scaling after applying a log-transform~\cite{platt1999probabilistic}, otherwise known as beta calibration~\cite{kull2017sigmoids}. Let $\sigma$ be the standard sigmoid function given by:
\[ \sigma(x) = \frac{1}{1 + \exp(-x)} \]
Then, our recalibration family $\mathcal{G}$ consists of $g$ parameterized by $a, c$, given by:
\[ g(z; a, c) = \sigma\Big( a\log{\frac{z}{1-z}} + c \Big) \]
In this set of synthetic experiments, we assume well-specification, that is $P(Y = 1 \mid Z=z) = g(z; a, c)$ for some $a, c$. We set $P(Z) = \mbox{Uniform}[0, 1]$. Since we know $P(Y = 1 \mid Z)$, we can approximate the true $\squaredce$ in this case, even for scaling methods. To do this, we sample $m=10000$ points $z_1, \dots, z_m$ independently from $P(Z)$. An \emph{unbiased} estimate of the $\squaredce$ then is:
\[ \squaredce(g) \approx \frac{1}{m} \sum_{i=1}^m \big[P(Y \mid Z = z_i) - g(z_i)\big]^2 \]
For each $n$ (number of recalibration samples) and $B$ (number of bins), we run either histogram binning or \ourcal{} with scaling family $\mathcal{G}$ and evaluate its calibration error as described above. We repeat this 1000 times, and compute 90\% confidence intervals. We fix $a = 2$ and $c = 1$.

\begin{figure}
  \centering
  \centering
     \begin{subfigure}[b]{0.48\textwidth}
         \centering
         \includegraphics[width=\textwidth]{hist_vary_n_a2_b1.png}
         \caption{Histogram binning.
         }
         \label{fig:well-spec-vary-n-hist}
     \end{subfigure}
     \hfill
     \begin{subfigure}[b]{0.48\textwidth}
         \centering
         \includegraphics[width=\textwidth]{var_red_vary_n_a2_b1.png}
         \caption{\Ourcal{}.
         }
         \label{fig:well-spec-vary-n-var-red}
     \end{subfigure}
  \caption{
    Plots of $1/\epsilon^2$ against $n$ (recall that $\epsilon^2$ is the $\lsquared$ calibration error). We see that for both methods $1/\epsilon^2$ increases approximately linearly with $n$, which match the theoretical bounds.
    \tnote{does it make sense to make this two plots have the same y limit? Or we could even put the two curves one the same figure?}
}
  \label{fig:well-spec-vary-n}
\end{figure}

In the first sub-experiment we fix $B = 10$ and vary $n$, plotting $1/\epsilon^2$ in Figure~\ref{fig:well-spec-vary-n} (recall that $\epsilon^2$ is the $\lsquared$ calibration error). We plot the calibration errors for each method in a different plot because of the difference in scales, \ourcal{} achieves a much lower calibration error than histogram binning. As the theory predicts, we see that $1/\epsilon^2$ is approximately linear in $n$ for both calibrators. For example, when $B=10$ if we increase from $n=1000$ to $n=2000$ the $\lsquared$ calibration error of histogram binning decreases by $2.00 \pm 0.06$ times, and the $\lsquared$ calibration error of our method decreases by $1.98 \pm 0.09$ times.

\begin{figure}
  \centering
  \centering
     \begin{subfigure}[b]{0.48\textwidth}
         \centering
         \includegraphics[width=\textwidth]{hist_vary_b_a2_b1.png}
         \caption{Histogram binning.
         }
         \label{fig:well-spec-vary-b-hist}
     \end{subfigure}
     \hfill
     \begin{subfigure}[b]{0.48\textwidth}
         \centering
         \includegraphics[width=\textwidth]{var_red_vary_b_a2_b1.png}
         \caption{\Ourcal{}.
         }
         \label{fig:well-spec-vary-b-var-red}
     \end{subfigure}
  \caption{
    Plots of $1/\epsilon^2$ against $b$ (recall that $\epsilon^2$ is the $\lsquared$ calibration error). Note that the $Y$ axis for \ourcal{} is clipped to 6600 and 7800 to show the relevant region. We see that for histogram binning $1/\epsilon^2$ scales close to $1/B$, in other words the calibration error increases with the number of bins (important note: the plot decreases because we plot the inverse $1/\epsilon^2$). For \ourcal{} $1/\epsilon^2$ is relatively constant, within the margin of estimation error, as predicted by the theory.
    \tnote{similar comments to the those for the figure above}
}
  \label{fig:well-spec-vary-b}
\end{figure}

In the second sub-experiment we fix $n = 2000$ and vary $B$, plotting $1/\epsilon^2$ in Figure~\ref{fig:well-spec-vary-b}. For \ourcal{} $1/\epsilon^2$ is nearly constant (within the margin of error), but for histogram binning $1/\epsilon^2$ scales close to $1/B$. When $n = 2000$ and we increase from $5$ to $20$ bins, our method's $\squaredce$ decreases by $2\% \pm 7\%$ but for histogram binning it increases by $3.71 \pm 0.15$ \emph{times}. For reference, we plot $P(Y \mid Z = z)$ in Figure~\ref{fig:well-spec-curve}.

\begin{figure}
  \centering
  \centering
     \begin{subfigure}[b]{0.48\textwidth}
         \centering
         \includegraphics[width=\textwidth]{curve_a2_b1.png}
         \caption{$P(Y \mid Z=z)$ for Experiment (A)
         }
         \label{fig:well-spec-curve}
     \end{subfigure}
     \hfill
     \begin{subfigure}[b]{0.48\textwidth}
         \centering
         \includegraphics[width=\textwidth]{noise_curve_a2_b1.png}
         \caption{$P(Y \mid Z=z)$ for Experiment (B)
         }
         \label{fig:noisy-spec-curve}
     \end{subfigure}
  \caption{
    Plots of $P(Y \mid Z=z)$ against $z$ for both synthetic experiments.
}
  \label{fig:p_y_z_plots}
\end{figure}


\textbf{(B) Synthetic experiments to compare \ourcal{} and the scaling method}: We run an illustrative toy experiment to show that there are some cases where \ourcal{} does better than the underlying scaling method---there are other cases where the underlying scaling method does better. \ourcal{} can do better because if we have infinite data, Proposition~\ref{prop:bin_low_bound} showed that the binned version $g_{\bins{}}$ has lower calibration error than $g$. On the other hand, in step 3 of \ourcal{} algorithm we empirically bin the outputs of the scaling method which incurs an estimation error, and could mean \ourcal{} has higher calibration error than the underlying scaling method. Our key advantage is that unlike scaling methods our method has measurable calibration error so if we are not calibrated we can get more data or use a different scaling family.

Building on the previous synthetic experiments, in this experiment, we set the ground truth $P(Y = 1 \mid Z=z) = g(z; a, c) + h(z)$ where for each $z$, $h(z) \in \{-0.02, 0.02\}$ with equal probability. In this case we set $P(Z) = \mbox{Uniform}[0.25, 0.75]$ so that $P(Y = 1 \mid Z=z) \in [0, 1]$. We fix $B=10$ and vary $n$, plotting the $\lsquared$ calibration error $\epsilon^2$ in Figure~\ref{fig:well-spec-vary-b}. With $B=10$ bins, $n = 3000$ the $\lsquared$ calibration error is $5.2 \pm 1.1$ times lower for \ourcal{} than the underlying scaling method using a sigmoid recalibrator. For reference, we plot $P(Y \mid Z = z)$ in Figure~\ref{fig:noisy-spec-curve}.

\begin{figure}
  \centering
  \includegraphics[width=0.6\textwidth]{noise_vary_n_a2_b1.png}
  \caption{Plot of $\epsilon^2$ ($\lsquared$ calibration error) against number of samples $n$ used to recalibrate. We can see in this case \ourcal{} consistently gets lower calibration error.
  }
  \label{fig:well-spec-vary-b}
\end{figure}


\pl{I hope these exist somewhere :-)}


\newpage

\newcommand{\lonece}[0]{\ensuremath{\ell_1\textup{-CE}}}
\newcommand{\loneerror}[0]{\ensuremath{\mathcal{E}}}
\newcommand{\pluginLoneEst}[0]{\ensuremath{\hat{\mathcal{E}}_{\textup{pl}}}}
\newcommand{\debiasedLoneEst}[0]{\ensuremath{\hat{\mathcal{E}}_{\textup{db}}}}

\section{Additional experiments for section~\ref{sec:verifying_calibration}}
\label{sec:verifying_calibration_appendix_experiments}

\subsection{Debiasing the ECE}
\label{sec:debiasing_ece_experiments}

We propose a way to more accurately estimate the $\ell_1$ calibration error (popularly known as ECE), and run experiments on ImageNet and CIFAR-10 that show that we can estimate the error much better than prior work, which uses the plugin estimator. The key insight is the same as for the $\ell_2$ calibration error---the plugin estimator for the $\ell_1$ calibration error is biased and this bias leads to inaccurate estimates. To estimate the error better we can subtract an approximation of the bias which leads to a better estimate. The main difference is that for the $\ell_1$ calibration error we were not able to approximate the bias with a closed form expression and instead use a Gaussian approximation.

Recall that the $\lonece{}$ is the $\ell_p{}$ calibration error with $p=1$, redefined below:

\begin{definition}
The $\ell_1$ calibration error of $f : \mathcal{X} \to [0, 1]$ is given by:
\begin{align}
\lonece(f) = \expect\big[ \left|f(X) - \expect[Y \mid f(X)] \right| \big]
\end{align}
\end{definition}

Estimating the $\lonece{}$ for many models is challenging (see Section~\ref{sec:challenges-measuring}) so prior work instead selects a binning scheme $\bins{}$ and estimates $\lonece(f_{\bins{}})$ of a model $f$. Suppose we wish to measure the binned calibration error $\loneerror{} = \lonece(f_{\bins{}})$ of a model $f : \mathcal{X} \to [0, 1]$ where $|\bins{}| = B$. Suppose we get an evaluation set $T_n = \{(x_1, y_1), \dots, (x_n, y_n)\}$. Past work typically estimates the $\ell_1$ calibration error using a plugin estimate for each term:

\begin{definition}[Plugin estimator for $\lonece{}$]
  Let $L_k$ denote the data points where the model outputs a prediction in the $k$-th bin of $\bins{}$: $L_k = \{ (x_j, y_j) \in T_n \; | \; f(x_j) \in I_k \}$.
  
  Let $\hat{p}_k$ be the estimated probability of $f$ outputting a prediction in the $k$-th bin:
$\hat{p}_k = \frac{|L_k|}{n}$.

Let $\hat y_k$ be the empirical average of $Y$ in the $k$-th bin: $\hat y_k = \sum_{x, y \in L_k} \frac{y}{|L_k|}$.

Let $\hat s_k$ be the empirical average of the model outputs in the $k$-th bin: $\hat s_k = \sum_{x, y \in L_k} \frac{f(x)}{|L_k|}$.
  
  The plugin estimate for the binned $\ell_1$ calibration error is the weighted squared difference between $\hat y_k$ and $\hat s_k$:
\[ \pluginLoneEst{} = \sum_{k=1}^B \hat{p}_k \lvert \hat s_k - \hat y_k \rvert \]
\end{definition}

The plugin estimate is a biased estimate of the binned calibration error. Intuitively, this is because of the absolute value: on any finite samples $s_k$ and $y_k$ will differ and the absolute value of the difference will be positive, even if the population values are the same. More concretely consider a model $f$ where $\expect[f(X) \mid f(X) \in I_k] = \expect[Y \mid f(X) \in I_k]$ in every bin $k$. In that case the binned calibration error $\loneerror{}$ is $0$. But on any finite samples the plugin estimate $\pluginLoneEst{}$ will be larger than $0$. In particular, the plugin estimator overestimates the binned calibration error, and the extent of overestimation may be different for different models.

To improve the estimate, we can subtract an approximation of the bias. That is, we would like to output $\pluginLoneEst{} - (\expect[\pluginLoneEst{}] - \loneerror{})$ as our estimate of the calibration error, where $\expect[\pluginLoneEst{}] - \loneerror{}$ is the bias. However, $\expect[\pluginLoneEst{}] - \loneerror{}$ is difficult to approximate in closed form. Instead, we propose approximating it by simulating draws from a normal approximation. More precisely, let $y_k = \expect[Y \mid f(X) \in I_k]$. Then, each label in the $k$-th bin is a draw from a Bernoulli distribution with parameter $y_k$. So $\hat y_k$ is the mean of $n \hat p_k$ Bernoulli draws. Assuming that the number of points in each bin is not too small, we can approximate $\hat y_k$ using a Gaussian approximation, and use that to approximate the bias $\expect[\pluginLoneEst{}] - \loneerror{}$.

\begin{definition}[Debiased estimator for $\lonece{}$]
For each $k$, let $R_k$ be a random variable sampled from a normal approximation of the label distribution in the $k$-th bin: $R_k \sim N(\hat y_k, \frac{\hat y_k (1 - \hat y_k)}{n \hat{p}_k})$. The debiased estimate for the binned $\ell_1$ calibration error subtracts an approximation of the bias from the plugin estimate:
\[ \debiasedLoneEst{} = \pluginLoneEst{} - (\expect\Big[\sum_{k=1}^B \hat{p}_k \lvert \hat s_k - R_k \rvert \Big] - \pluginLoneEst{}) \]
\end{definition}

We can approximate the expectation in the debiased estimator by simulating many draws from a normal distribution, which is computationally fairly inexpensive. Note that our proposed estimator is a heuristic approach, and future work should examine whether we can get provably better estimation rates for estimating the $\ell_1$ calibration error, as we did for the $\ell_2$ calibration error. That might involve analyzing our proposed estimator, or may involve coming up with a completely different estimator.

\begin{figure}
  \centering
  \centering
     \begin{subfigure}[b]{0.45\textwidth}
         \centering
         \includegraphics[width=\textwidth]{images/imnet_scaling_ece_curve_15.png}
         \caption{$B = 15$
         }
     \end{subfigure}
     \hfill
     \begin{subfigure}[b]{0.45\textwidth}
         \centering
         \includegraphics[width=\textwidth]{images/imnet_scaling_ece_curve_100.png}
         \caption{$B = 100$
         }
     \end{subfigure}
  \caption{
    Mean-squared errors of plugin and debiased estimators on a recalibrated VGG16 model on ImageNet with $90\%$ confidence intervals (lower values better). The debiased estimator is closer to the ground truth, which corresponds to $0$ on the vertical axis, and much more so when $B$ is large or $n$ is small.
    Note that this is the MSE of the ECE estimates, not the MSE of the model in Figure~\ref{fig:nan2}.
}
  \label{fig:mse_estimators_imagenet_ece_bins}
\end{figure}

We run multiclass top-label calibration experiment on CIFAR-10 and ImageNet which suggests that the debiased estimator produces better estimates of the calibration error than the plugin estimator. We describe the protocol for ImageNet first, which is similar to the experimental protocol in Section~\ref{sec:verifying_calibration_experiments}. We split the validation set of size 50,000 into two sets $\calset{}$ and $\verifset{}$ of sizes 3,000 and 47,000 respectively. Note that a practitioner would not need so many data points when estimating their model's calibration, we use 47,000 points only so that we can reliably compare the estimators. We use $\calset{}$ to re-calibrate a trained VGG-16 model and select a binning scheme $\bins{}$ so that each bin contains an equal number of points (uniform-mass binning). We calibrate the top probability prediction as described in Section~\ref{sec:formulation} using Platt scaling. For varying values of $n$, we sample $n$ points with replacement from $\verifset{}$, and estimate the binned $\ell_1$ calibration error (ECE) using the plugin estimator and our proposed debiased estimator. We used $B = 100$ or $B = 15$ bins in our experiments. We then compute the squared deviation of these estimates from the binned $\ell_1$ calibration error measured on the entire set $\verifset{}$. We repeat this resampling 1,000 times to get the mean squared deviation of the estimates from the ground truth and 90\% confidence intervals. Figure~\ref{fig:mse_estimators_imagenet_ece_bins} shows that the debiased estimates are much closer to the ground truth than the plugin estimates---the difference is especially significant when the number of samples $n$ is small or the number of bins $B$ is large. Note that having a perfect estimate corresponds to $0$ on the vertical axis.

\begin{figure}
  \centering
  \centering
     \begin{subfigure}[b]{0.45\textwidth}
         \centering
         \includegraphics[width=\textwidth]{images/cifar_scaling_ece_curve_15.png}
         \caption{$B = 15$
         }
     \end{subfigure}
     \hfill
     \begin{subfigure}[b]{0.45\textwidth}
         \centering
         \includegraphics[width=\textwidth]{images/cifar_scaling_ece_curve_100.png}
         \caption{$B = 100$
         }
     \end{subfigure}
  \caption{
    Mean-squared errors of plugin and debiased estimators on a recalibrated VGG16 model on CIFAR-10 with $90\%$ confidence intervals (lower values better). The debiased estimator is closer to the ground truth, which corresponds to $0$ on the vertical axis, especially when $B$ is large or $n$ is small.
    Note that this is the MSE of the ECE estimates, not the MSE of the model in Figure~\ref{fig:nan2}.
}
  \label{fig:mse_estimators_cifar_ece_bins}
\end{figure}

For CIFAR-10 we use the same protocol except we split the validation set of size 10,000 into two sets $\calset{}$ and $\verifset{}$ of sizes 3,000 and 7,000 respectively. Figure~\ref{fig:mse_estimators} shows that the debiased estimates are much closer to the ground truth than the plugin estimates in this case as well.


\subsection{Additional experiments for estimating calibration error}

In Section~\ref{sec:verifying_calibration} we ran an experiment on CIFAR-10 to show that the debiased estimator gives estimates closer to the true calubration error than the plugin estimator. To give more insight into this, Figure~\ref{fig:histograms_estimators_bins} shows a histogram of the absolute difference between the estimates and ground truth for the plugin and debiased estimator, over the 1,000 resamples, when we use $B = 10$ or $B = 100$ bins. For $B = 10$ bins it is not completely clear which estimator is doing better but the debiased estimator avoids very bad estimates. However, when $B = 100$, the debiased estimator produces estimates much closer to the ground truth ($0$ on the x-axis).

\begin{figure}
  \centering
  \centering
     \begin{subfigure}[b]{0.45\textwidth}
         \centering
         \includegraphics[width=\textwidth]{histogram_estimators_10_bins.png}
         \caption{$B = 10$ bins}
     \end{subfigure}
     \hfill
     \begin{subfigure}[b]{0.45\textwidth}
         \centering
         \includegraphics[width=\textwidth]{histogram_estimates_100_bins.png}
         \caption{$B = 100$ bins}
     \end{subfigure}
  \caption{Histograms of the absolute value of the difference between estimated and ground truth squared calibration errors ($0$ on the x-axis). For $B = 10$ bins, the results are mixed but we avoid very bad estimates. For $B=100$ our estimates are much closer to ground truth.\tnote{same comments as before}}
  \label{fig:histograms_estimators_bins}
\end{figure}

We also show histograms for the ECE experiments in Appendix~\ref{sec:debiasing_ece_experiments}, in Figure~\ref{fig:histograms_estimators_ece_imagenet_bins} for ImageNet and Figure~\ref{fig:histograms_estimators_ece_cifar_10_bins} for CIFAR-10. These histograms show that the proposed debiased estimator produces estimates much closer to the ground truth than the plugin estimator.

\begin{figure}
  \centering
  \centering
     \begin{subfigure}[b]{0.45\textwidth}
         \centering
         \includegraphics[width=\textwidth]{imnet_scaling_ece_histogram_15.png}
         \caption{$B = 15$ bins}
     \end{subfigure}
     \hfill
     \begin{subfigure}[b]{0.45\textwidth}
         \centering
         \includegraphics[width=\textwidth]{imnet_scaling_ece_histogram_100.png}
         \caption{$B = 100$ bins}
     \end{subfigure}
  \caption{Histograms of the absolute value of the difference between estimated and ground truth ECE ($0$ on the x-axis) on ImageNet.}
  \label{fig:histograms_estimators_ece_imagenet_bins}
\end{figure}

\begin{figure}
  \centering
  \centering
     \begin{subfigure}[b]{0.45\textwidth}
         \centering
         \includegraphics[width=\textwidth]{cifar_scaling_ece_histogram_15.png}
         \caption{$B = 15$ bins}
     \end{subfigure}
     \hfill
     \begin{subfigure}[b]{0.45\textwidth}
         \centering
         \includegraphics[width=\textwidth]{cifar_scaling_ece_histogram_100.png}
         \caption{$B = 100$ bins}
     \end{subfigure}
  \caption{Histograms of the absolute value of the difference between estimated and ground truth ECE ($0$ on the x-axis) on CIFAR-10.}
  \label{fig:histograms_estimators_ece_cifar_10_bins}
\end{figure}

We also ran a marginal multiclass calibration experiment on CIFAR-10 to show that our estimator allows us to select models with a lower mean-squared error subject to a given calibration constraint. In this case we split the validation set into $\calset{}$ and $\verifset{}$ of size 6000 and 4000 respectively, and recalibrated a trained model on $\calset{}$. On $\verifset{}$, we estimate the calibration error using the plugin and debiased estimators and use 100 Bootstrap resamples to compute a 90\% upper confidence bound on the estimate (from the variance of the Bootstrap samples). We compute the mean-squared error and the upper bounds on the calibration error for $B = 10, 15, \dots, 100$ and show the Pareto curve in Figure~\ref{fig:mse_vs_ce_estimator}. Figure~\ref{fig:mse_vs_ce_estimator} shows that for any desired calibration error, the debiased estimator enables us to pick out models with a better mean-squared error. For example, if we want a model with calibration error less than $1.5\%$, the debiased estimator tells us we can confidently use 100 bins, while relying on the plugin estimator only lets us use 15 bins and incurs a 13\% higher mean-squared error.

% Note that in our theoretical results in Section~\ref{sec:verifying_calibration}, we focused on hypothesis testing, using the estimator to test whether a model has calibration error $\leq \epsilon$ or not. However, the 

% \pl{say explicitly that we never say 'not calibrated'; there's a disconnect between the theory, which requires $r$ and $\epsilon$ and what we're doing here}



\begin{figure}
  \centering
  \includegraphics[width=0.9\textwidth]{mse_vs_verified_error_plugin_vs_ours.png}
  \caption{Plot of mean-squared error against 90\% upper bounds on the calibration error computed by the debiased estimator and the plugin estimator, when we vary the number of bins $B$. For a given calibration error, our estimator enables us to choose models with a better mean-squared error. If we want a model with calibration error less than 0.015, the debiased estimator tells us we can confidently use 100 bins, while relying on the plugin estimator only lets us use 15 bins and incurs a 13\% higher mean-squared error.}
  \label{fig:mse_vs_ce_estimator}
\end{figure}



\end{document}
